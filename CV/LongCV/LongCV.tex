%!TEX TS-program = xelatex
%!TEX encoding = UTF-8 Unicode
% Awesome CV LaTeX Template for CV/Resume
%
% This template has been downloaded from:
% https://github.com/posquit0/Awesome-CV
%
% Author:
% Claud D. Park <posquit0.bj@gmail.com>
% http://www.posquit0.com
%
%
% Adapted to be an Rmarkdown template by Mitchell O'Hara-Wild
% 23 November 2018
%
% Template license:
% CC BY-SA 4.0 (https://creativecommons.org/licenses/by-sa/4.0/)
%
%-------------------------------------------------------------------------------
% CONFIGURATIONS
%-------------------------------------------------------------------------------
% A4 paper size by default, use 'letterpaper' for US letter
\documentclass[11pt, a4paper]{awesome-cv}

% Configure page margins with geometry
\geometry{left=1.4cm, top=.8cm, right=1.4cm, bottom=1.8cm, footskip=.5cm}

% Specify the location of the included fonts
\fontdir[fonts/]

% Color for highlights
% Awesome Colors: awesome-emerald, awesome-skyblue, awesome-red, awesome-pink, awesome-orange
%                 awesome-nephritis, awesome-concrete, awesome-darknight

\colorlet{awesome}{awesome-red}

% Colors for text
% Uncomment if you would like to specify your own color
% \definecolor{darktext}{HTML}{414141}
% \definecolor{text}{HTML}{333333}
% \definecolor{graytext}{HTML}{5D5D5D}
% \definecolor{lighttext}{HTML}{999999}

% Set false if you don't want to highlight section with awesome color
\setbool{acvSectionColorHighlight}{true}

% If you would like to change the social information separator from a pipe (|) to something else
\renewcommand{\acvHeaderSocialSep}{\quad\textbar\quad}

\def\endfirstpage{\newpage}

%-------------------------------------------------------------------------------
%	PERSONAL INFORMATION
%	Comment any of the lines below if they are not required
%-------------------------------------------------------------------------------
% Available options: circle|rectangle,edge/noedge,left/right

\name{Sarah}{Gaichas}

\position{Research Fishery Biologist}
\address{NOAA NMFS Northeast Fisheries Science Center}

\mobile{+1 508 495 2016}
\email{\href{mailto:Sarah.Gaichas@noaa.gov}{\nolinkurl{Sarah.Gaichas@noaa.gov}}}
\github{sgaichas}

% \gitlab{gitlab-id}
% \stackoverflow{SO-id}{SO-name}
% \skype{skype-id}
% \reddit{reddit-id}


\usepackage{booktabs}

% Templates for detailed entries
% Arguments: what when with where why
\usepackage{etoolbox}
\def\detaileditem#1#2#3#4#5{%
\cventry{#1}{#3}{#4}{#2}{\ifx#5\empty\else{\begin{cvitems}#5\end{cvitems}}\fi}\ifx#5\empty{\vspace{-4.0mm}}\else\fi}
\def\detailedsection#1{\begin{cventries}#1\end{cventries}}

% Templates for brief entries
% Arguments: what when with
\def\briefitem#1#2#3{\cvhonor{}{#1}{#3}{#2}}
\def\briefsection#1{\begin{cvhonors}#1\end{cvhonors}}

\providecommand{\tightlist}{%
	\setlength{\itemsep}{0pt}\setlength{\parskip}{0pt}}

%------------------------------------------------------------------------------



\begin{document}

% Print the header with above personal informations
% Give optional argument to change alignment(C: center, L: left, R: right)
\makecvheader

% Print the footer with 3 arguments(<left>, <center>, <right>)
% Leave any of these blank if they are not needed
% 2019-02-14 Chris Umphlett - add flexibility to the document name in footer, rather than have it be static Curriculum Vitae
\makecvfooter
  {December 2020}
    {Sarah Gaichas~~~·~~~Curriculum Vitae}
  {\thepage}


%-------------------------------------------------------------------------------
%	CV/RESUME CONTENT
%	Each section is imported separately, open each file in turn to modify content
%------------------------------------------------------------------------------



\hypertarget{education}{%
\section{Education}\label{education}}

\detailedsection{\detaileditem{PhD}{2000 - 2006}{University of Washington}{Seattle}{\empty}\detaileditem{M.S.}{1994 - 1997}{College of William and Mary, School of Marine Science}{Gloucester Point}{\empty}\detaileditem{B.A.}{1986 - 1991}{Swarthmore College}{Swarthmore}{\empty}}

\hypertarget{graduate-supervision-and-examination}{%
\section{Graduate Supervision and Examination}\label{graduate-supervision-and-examination}}

\textbf{Adjunct professor} \emph{University of Massachusetts, Dartmouth}

One current UMassD PhD. student committee.\\
One current Ph.D.~student committee at the College of William and Mary (Virginia Institute of Marine Science).

2020 External examiner for a PhD student at University of Iceland\\
2019 (Exam) PhD student committee at the University of Massachusetts, Dartmouth\\
2019 External examiner for a PhD student at Dalhousie University\\
2016 External examiner for a PhD student at the University of Cape Town\\
2016 (Exam) PhD student committee at the University of Washington\\
2015 (Exam) PhD student committee at Oregon State University

\hypertarget{experience}{%
\section{Experience}\label{experience}}

\textbf{Research Fishery Biologist} \emph{NOAA/NMFS Northeast Fisheries Sci. Center, Woods Hole, MA}

\begin{verbatim}
Ecosystem Assessment, MSE       September 2011 - present
\end{verbatim}

Contribute to integrated ecosystem assessments and management strategy evaluations (MSEs) for the Northeast U.S. continental shelf by analyzing research survey, fishery, oceanographic, and economic data, and by developing and testing multispecies assessment models and ecosystem indicators. Conduct research on ecosystem based management reference points balancing multiple objectives. Chair National and Northeast MSE working groups to build capacity and coordinate work plans. Provide results in peer-reviewed literature and to the New England and Mid-Atlantic Fishery Management Councils.

\textbf{Research Fishery Biologist} \emph{NOAA/NMFS Alaska Fisheries Science Center, Seattle, WA}

\begin{verbatim}
Ecosystem Modeling          June 2006 – September 2011
\end{verbatim}

Analyzed food web, fishery, and research survey data and developed ecosystem models to evaluate the cumulative effects of fishing, climate change, and other factors on the marine ecosystem.

\begin{verbatim}
Stock Assessment            November 1998 – June 2006
\end{verbatim}

Assessed the condition of commercially exploited fish stocks through analysis of fishery and research survey data combined with population dynamics modeling.

\begin{verbatim}
Observer Program            March 1997 – October 1999
\end{verbatim}

Conducted statistical analyses of fishery observer data, evaluated present and proposed data collection methods through directed research, and examined uses of observer data in multiple applications.

\hypertarget{research-science-and-management-organizations}{%
\section{Research, Science, and Management Organizations}\label{research-science-and-management-organizations}}

\begin{itemize}
\tightlist
\item
  Mid-Atlantic FMC Scientific and Statistical Committee, (2014-present); determining Allowable Biological Catch (ABC) advice for fishery management and other scientific support.
\item
  Member and past co-chair of the ICES Working Group on North Atlantic Regional Seas; developing ecosystem indicators and integrated ecosystem assessments for Canadian and U.S. waters.
\item
  Co-Chair of the ICES Working Group on Multispecies Assessment Methods: reviewing and contributing to multispecies and ecosystem modeling progress across ICES regions.
\item
  Advisory partner in the European Commission's PANDORA project (2018-2022), which is designing toolsets for new dynamic ocean resource assessments and exploitation.
\end{itemize}

\hypertarget{publications}{%
\section{Publications}\label{publications}}

\protect\hyperlink{cite-lucey_evaluating_2021}{Lucey, S. M, K. Y. Aydin, S. K.
Gaichas, S. X. Cadrin, G. Fay, M. J. Fogarty, and A.
Punt} (2021). ``Evaluating fishery
management strategies using an ecosystem model as an operating model''.
In: \emph{Fisheries Research} 234, p.~105780. DOI:
\href{https://doi.org/10.1016\%2Fj.fishres.2020.105780}{10.1016/j.fishres.2020.105780}.

\protect\hyperlink{cite-bastille_improving_2020}{Bastille, K, S. Hardison, L.
deWitt, J. Brown, J. Samhouri, S. Gaichas, S. Lucey, K. Kearney, B.
Best, S. Cross, S. Large, and E.
Spooner} (2020). ``Improving the IEA
Approach Using Principles of Open Data Science''. In: \emph{Coastal
Management} 0.0, pp.~1-18. DOI:
\href{https://doi.org/10.1080\%2F08920753.2021.1846155}{10.1080/08920753.2021.1846155}.

\protect\hyperlink{cite-link_changing_2020}{Link, J. S, G. Huse, S. Gaichas,
and A. R. Marshak} (2020). ``Changing how we
approach fisheries: A first attempt at an operational framework for
ecosystem approaches to fisheries management''. In: \emph{Fish and Fisheries}
21.2, pp.~393-434. DOI:
\href{https://doi.org/10.1111\%2Ffaf.12438}{10.1111/faf.12438}.

\protect\hyperlink{cite-lucey_conducting_2020}{Lucey, S. M, S. K. Gaichas, and
K. Y. Aydin} (2020). ``Conducting
reproducible ecosystem modeling using the open source mass balance
model Rpath''. In: \emph{Ecological Modelling} 427, p.~109057. DOI:
\href{https://doi.org/10.1016\%2Fj.ecolmodel.2020.109057}{10.1016/j.ecolmodel.2020.109057}.

\protect\hyperlink{cite-muffley_there_2020}{Muffley, B, S. Gaichas, G. DePiper,
R. Seagraves, and S. Lucey} (2020). ``There Is
no I in EAFM Adapting Integrated Ecosystem Assessment for Mid-Atlantic
Fisheries Management''. In: \emph{Coastal Management} 0.0, pp.~1-17. DOI:
\href{https://doi.org/10.1080\%2F08920753.2021.1846156}{10.1080/08920753.2021.1846156}.

\protect\hyperlink{cite-weiskopf_climate_2020}{Weiskopf, S. R, M. A.
Rubenstein, L. G. Crozier, S. Gaichas, R. Griffis, J. E. Halofsky, K.
J. W. Hyde, T. L. Morelli, J. T. Morisette, R. C. Muñoz, A. J.
Pershing, D. L. Peterson, R. Poudel, M. D. Staudinger, A. E.
Sutton-Grier, L. Thompson, J. Vose, J. F. Weltzin, and K. P.
Whyte} (2020). ``Climate change effects on
biodiversity, ecosystems, ecosystem services, and natural resource
management in the United States''. In: \emph{Science of The Total
Environment}, p.~137782. DOI:
\href{https://doi.org/10.1016\%2Fj.scitotenv.2020.137782}{10.1016/j.scitotenv.2020.137782}.

\protect\hyperlink{cite-goethel_recent_2019}{Goethel, D. R, S. M. Lucey, A. M.
Berger, S. K. Gaichas, M. A. Karp, P. D. Lynch, and J. F.
Walter} (2019). ``Recent advances in
management strategy evaluation: introduction to the special issue
``Under pressure: addressing fisheries challenges with Management
Strategy Evaluation''". In: \emph{Canadian Journal of Fisheries and Aquatic
Sciences} 76.10, pp.~1689-1696. DOI:
\href{https://doi.org/10.1139\%2Fcjfas-2019-0084}{10.1139/cjfas-2019-0084}.

\protect\hyperlink{cite-karp_accounting_2019}{Karp, M. A, J. O. Peterson, P. D.
Lynch, R. B. Griffis, C. F. Adams, W. S. Arnold, L. A. K. Barnett, Y.
deReynier, J. DiCosimo, K. H. Fenske, S. K. Gaichas, A. Hollowed, K.
Holsman, M. Karnauskas, D. Kobayashi, A. Leising, J. P. Manderson, M.
McClure, W. E. Morrison, E. Schnettler, A. Thompson, J. T. Thorson, J.
F. Walter, A. J. Yau, R. D. Methot, and J. S.
Link} (2019). ``Accounting for shifting
distributions and changing productivity in the development of
scientific advice for fishery management''. In: \emph{ICES Journal of Marine
Science} 76.5, pp.~1305-1315. DOI:
\href{https://doi.org/10.1093\%2Ficesjms\%2Ffsz048}{10.1093/icesjms/fsz048}.

\protect\hyperlink{cite-townsend_progress_2019}{Townsend, H, C. J. Harvey, Y.
deReynier, D. Davis, S. G. Zador, S. Gaichas, M. Weijerman, E. L.
Hazen, and I. C. Kaplan} (2019).
``Progress on Implementing Ecosystem-Based Fisheries Management in the
United States Through the Use of Ecosystem Models and Analysis''. In:
\emph{Frontiers in Marine Science} 6. DOI:
\href{https://doi.org/10.3389\%2Ffmars.2019.00641}{10.3389/fmars.2019.00641}.

\protect\hyperlink{cite-deroba_dream_2018}{Deroba, J. J, S. K. Gaichas, M. Lee,
R. G. Feeney, D. V. Boelke, and B. J. Irwin}
(2018). ``The dream and the reality: meeting decision-making time frames
while incorporating ecosystem and economic models into management
strategy evaluation''. In: \emph{Canadian Journal of Fisheries and Aquatic
Sciences}. DOI:
\href{https://doi.org/10.1139\%2Fcjfas-2018-0128}{10.1139/cjfas-2018-0128}.

\protect\hyperlink{cite-feeney_integrating_2018}{Feeney, R. G, D. V. Boelke, J.
J. Deroba, S. K. Gaichas, B. J. Irwin, and M.
Lee} (2018). ``Integrating Management
Strategy Evaluation into fisheries management: advancing best practices
for stakeholder inclusion based on an MSE for Northeast U.S. Atlantic
herring''. In: \emph{Canadian Journal of Fisheries and Aquatic Sciences}.
DOI:
\href{https://doi.org/10.1139\%2Fcjfas-2018-0125}{10.1139/cjfas-2018-0125}.

\protect\hyperlink{cite-gaichas_implementing_2018}{Gaichas, S. K, G. S.
DePiper, R. J. Seagraves, B. W. Muffley, M. Sabo, L. L. Colburn, and A.
L. Loftus} (2018). ``Implementing
Ecosystem Approaches to Fishery Management: Risk Assessment in the US
Mid-Atlantic''. In: \emph{Frontiers in Marine Science} 5. DOI:
\href{https://doi.org/10.3389\%2Ffmars.2018.00442}{10.3389/fmars.2018.00442}.

\protect\hyperlink{cite-goethel_closing_2018}{Goethel, D. R, S. M. Lucey, A. M.
Berger, S. K. Gaichas, M. A. Karp, P. D. Lynch, J. F. Walter III, J. J.
Deroba, S. K. Miller, and M. J. Wilberg}
(2018). ``Closing the Feedback Loop: On Stakeholder Participation in
Management Strategy Evaluation''. In: \emph{Canadian Journal of Fisheries and
Aquatic Sciences}. DOI:
\href{https://doi.org/10.1139\%2Fcjfas-2018-0162}{10.1139/cjfas-2018-0162}.

\protect\hyperlink{cite-olsen_ocean_2018}{Olsen, E, I. C. Kaplan, C. Ainsworth,
G. Fay, S. Gaichas, R. Gamble, R. Girardin, C. H. Eide, T. F. Ihde, H.
N. Morzaria-Luna, K. F. Johnson, M. Savina-Rolland, H. Townsend, M.
Weijerman, E. A. Fulton, and J. S. Link}
(2018). ``Ocean Futures Under Ocean Acidification, Marine Protection,
and Changing Fishing Pressures Explored Using a Worldwide Suite of
Ecosystem Models''. In: \emph{Frontiers in Marine Science} 5. DOI:
\href{https://doi.org/10.3389\%2Ffmars.2018.00064}{10.3389/fmars.2018.00064}.

\protect\hyperlink{cite-depiper_operationalizing_2017}{DePiper, G. S, S. K.
Gaichas, S. M. Lucey, P. Pinto da Silva, M. R. Anderson, H. Breeze, A.
Bundy, P. M. Clay, G. Fay, R. J. Gamble, R. S. Gregory, P. S.
Fratantoni, C. L. Johnson, M. Koen-Alonso, K. M. Kleisner, J. Olson, C.
T. Perretti, P. Pepin, F. Phelan, V. S. Saba, L. A. Smith, J. C. Tam,
N. D. Templeman, and R. P.
Wildermuth} (2017).
``Operationalizing integrated ecosystem assessments within a
multidisciplinary team: lessons learned from a worked example''. In:
\emph{ICES Journal of Marine Science} 74.8, pp.~2076-2086. DOI:
\href{https://doi.org/10.1093\%2Ficesjms\%2Ffsx038}{10.1093/icesjms/fsx038}.

\protect\hyperlink{cite-gaichas_combining_2017}{Gaichas, S. K, M. Fogarty, G.
Fay, R. Gamble, S. Lucey, and L. Smith}
(2017). ``Combining stock, multispecies, and ecosystem level fishery
objectives within an operational management procedure: simulations to
start the conversation''. In: \emph{ICES Journal of Marine Science} 74.2, pp.
552-565. DOI:
\href{https://doi.org/10.1093\%2Ficesjms\%2Ffsw119}{10.1093/icesjms/fsw119}.

\protect\hyperlink{cite-holsman_ecosystem-based_2017}{Holsman, K, J. Samhouri,
G. Cook, E. Hazen, E. Olsen, M. Dillard, S. Kasperski, S. Gaichas, C.
R. Kelble, M. Fogarty, and K.
Andrews} (2017). ``An
ecosystem-based approach to marine risk assessment''. In: \emph{Ecosystem
Health and Sustainability} 3.1, p.~e01256. DOI:
\href{https://doi.org/10.1002\%2Fehs2.1256}{10.1002/ehs2.1256}.

\protect\hyperlink{cite-tommasi_managing_2017}{Tommasi, D, C. A. Stock, A. J.
Hobday, R. Methot, I. C. Kaplan, J. P. Eveson, K. Holsman, T. J.
Miller, S. Gaichas, M. Gehlen, A. Pershing, G. A. Vecchi, R. Msadek, T.
Delworth, C. M. Eakin, M. A. Haltuch, R. Séférian, C. M. Spillman, J.
R. Hartog, S. Siedlecki, J. F. Samhouri, B. Muhling, R. G. Asch, M. L.
Pinsky, V. S. Saba, S. B. Kapnick, C. F. Gaitan, R. R. Rykaczewski, M.
A. Alexander, Y. Xue, K. V. Pegion, P. Lynch, M. R. Payne, T.
Kristiansen, P. Lehodey, and F. E. Werner}
(2017). ``Managing living marine resources in a dynamic environment: The
role of seasonal to decadal climate forecasts''. In: \emph{Progress in
Oceanography} 152, pp.~15-49. DOI:
\href{https://doi.org/10.1016\%2Fj.pocean.2016.12.011}{10.1016/j.pocean.2016.12.011}.

\protect\hyperlink{cite-wildermuth_structural_2017}{Wildermuth, R. P, G. Fay,
and S. Gaichas} (2017). ``Structural
uncertainty in qualitative models for ecosystem-based management of
Georges Bank''. In: \emph{Canadian Journal of Fisheries and Aquatic Sciences}
75.10, pp.~1635-1643. DOI:
\href{https://doi.org/10.1139\%2Fcjfas-2017-0149}{10.1139/cjfas-2017-0149}.

\protect\hyperlink{cite-zador_linking_2017}{Zador, S. G, S. K. Gaichas, S.
Kasperski, C. L. Ward, R. E. Blake, N. C. Ban, A. Himes-Cornell, and J.
Z. Koehn} (2017). ``Linking ecosystem
processes to communities of practice through commercially fished
species in the Gulf of Alaska''. In: \emph{ICES Journal of Marine Science}
74.7, pp.~2024-2033. DOI:
\href{https://doi.org/10.1093\%2Ficesjms\%2Ffsx054}{10.1093/icesjms/fsx054}.

\protect\hyperlink{cite-gaichas_framework_2016}{Gaichas, S. K, R. J. Seagraves,
J. M. Coakley, G. S. DePiper, V. G. Guida, J. A. Hare, P. J. Rago, and
M. J. Wilberg} (2016). ``A Framework for
Incorporating Species, Fleet, Habitat, and Climate Interactions into
Fishery Management''. In: \emph{Frontiers in Marine Science} 3. DOI:
\href{https://doi.org/10.3389\%2Ffmars.2016.00105}{10.3389/fmars.2016.00105}.

\protect\hyperlink{cite-olsen_ecosystem_2016}{Olsen, E, G. Fay, S. Gaichas, R.
Gamble, S. Lucey, and J. S. Link} (2016).
``Ecosystem Model Skill Assessment. Yes We Can!'' In: \emph{PLOS ONE} 11.1.
Ed. by C. N. Bianchi, p.~e0146467. DOI:
\href{https://doi.org/10.1371\%2Fjournal.pone.0146467}{10.1371/journal.pone.0146467}.

\protect\hyperlink{cite-weijerman_atlantis_2016}{Weijerman, M, J. Link, E.
Fulton, E. Olsen, H. Townsend, S. Gaichas, C. Hansen, M.
Skern-Mauritzen, I. Kaplan, R. Gamble, G. Fay, M. Savina, C. Ainsworth,
I. van Putten, R. Gorton, R. Brainard, K. Larsen, and T.
Hutton} (2016). ``Atlantis Ecosystem
Model Summit: Report from a workshop''. In: \emph{Ecological Modelling} 335,
pp.~35-38. DOI:
\href{https://doi.org/10.1016\%2Fj.ecolmodel.2016.05.007}{10.1016/j.ecolmodel.2016.05.007}.

\protect\hyperlink{cite-zador_ecosystem_2016}{Zador, S. G, K. K. Holsman, K. Y.
Aydin, and S. K. Gaichas} (2016).
``Ecosystem considerations in Alaska: the value of qualitative
assessments''. In: \emph{ICES Journal of Marine Science: Journal du Conseil},
p.~fsw144. DOI:
\href{https://doi.org/10.1093\%2Ficesjms\%2Ffsw144}{10.1093/icesjms/fsw144}.

\protect\hyperlink{cite-gaichas_wasp_2015}{Gaichas, S, K. Aydin, and R. C.
Francis} (2015). ``Wasp waist or beer belly?
Modeling food web structure and energetic control in Alaskan marine
ecosystems, with implications for fishing and environmental forcing''.
In: \emph{Progress in Oceanography} 138, Part A, pp.~1-17. DOI:
\href{https://doi.org/10.1016\%2Fj.pocean.2015.09.010}{10.1016/j.pocean.2015.09.010}.

\protect\hyperlink{cite-smith_simulations_2015}{Smith, L, R. Gamble, S.
Gaichas, and J. Link} (2015).
``Simulations to evaluate management trade-offs among marine mammal
consumption needs, commercial fishing fleets and finfish biomass''. In:
\emph{Marine Ecology Progress Series} 523, pp.~215-232. DOI:
\href{https://doi.org/10.3354\%2Fmeps11129}{10.3354/meps11129}.

\protect\hyperlink{cite-gaichas_risk-based_2014}{Gaichas, S. K, J. S. Link, and
J. A. Hare} (2014). ``A risk-based
approach to evaluating northeast US fish community vulnerability to
climate change''. In: \emph{ICES Journal of Marine Science} 71.8, pp.
2323-2342. DOI:
\href{https://doi.org/10.1093\%2Ficesjms\%2Ffsu048}{10.1093/icesjms/fsu048}.

\protect\hyperlink{cite-ruzicka_dividing_2013}{Ruzicka, J. J, J. H. Steele, T.
Ballerini, S. K. Gaichas, and D. G.
Ainley} (2013). ``Dividing up the pie:
Whales, fish, and humans as competitors''. In: \emph{Progress in
Oceanography} 116, pp.~207-219. DOI:
\href{https://doi.org/10.1016\%2Fj.pocean.2013.07.009}{10.1016/j.pocean.2013.07.009}.

\protect\hyperlink{cite-ruzicka_analysis_2013}{Ruzicka, J, J. Steele, S.
Gaichas, T. Ballerini, D. Gifford, R. Brodeur, and E.
Hofmann} (2013). ``Analysis of Energy Flow
in US GLOBEC Ecosystems Using End-to-End Models''. In: \emph{Oceanography}
26.4, pp.~82-97. DOI:
\href{https://doi.org/10.5670\%2Foceanog.2013.77}{10.5670/oceanog.2013.77}.

\protect\hyperlink{cite-fu_relative_2012}{Fu, C, S. Gaichas, J. Link, A. Bundy,
J. Boldt, A. Cook, R. Gamble, K. Rong Utne, H. Liu, and K.
Friedland} (2012). ``Relative importance of
fisheries, trophodynamic and environmental drivers in a series of
marine ecosystems''. In: \emph{Marine Ecology Progress Series} 459, pp.
169-184. DOI: \href{https://doi.org/10.3354\%2Fmeps09805}{10.3354/meps09805}.

\protect\hyperlink{cite-gaichas_beyond_2012}{Gaichas, S. K, G. Odell, K. Y.
Aydin, and R. C. Francis} (2012). ``Beyond
the defaults: functional response parameter space and ecosystem-level
fishing thresholds in dynamic food web model simulations''. In:
\emph{Canadian Journal of Fisheries and Aquatic Sciences} 69, pp.~2077-2094.
DOI: \href{https://doi.org/10.1139\%2FF2012-099}{10.1139/F2012-099}.

\protect\hyperlink{cite-gaichas_what_2012}{Gaichas, S, A. Bundy, T. Miller, E.
Moksness, and K. Stergiou} (2012). ``What
drives marine fisheries production?'' In: \emph{Marine Ecology Progress
Series} 459, pp.~159-163. DOI:
\href{https://doi.org/10.3354\%2Fmeps09841}{10.3354/meps09841}.

\protect\hyperlink{cite-gaichas_assembly_2012}{Gaichas, S, R. Gamble, M. J.
Fogarty, H. Benoit, T. Essington, C. Fu, M. Koen-Alonso, and J.
Link} (2012). ``Assembly Rules for
Aggregate-Species Production Models: Simulations in Support of
Management Strategy Evaluation''. In: \emph{Marine Ecology Progress Series}
459, pp.~275-292.

\protect\hyperlink{cite-link_synthesizing_2012}{Link, J, S. Gaichas, T. Miller,
T. Essington, A. Bundy, J. Boldt, K. Drinkwater, and E.
Moksness} (2012). ``Synthesizing Lessons
Learned from Comparing Fisheries Production in 13 Northern Hemisphere
Ecosystems: Emergent Fundamental Features''. In: \emph{Marine Ecology
Progress Series} 459, pp.~293-302.

\protect\hyperlink{cite-link_dealing_2012}{Link, J, T. Ihde, C. Harvey, S.
Gaichas, J. Field, J. Brodziak, H. Townsend, and R.
Peterman} (2012). ``Dealing with uncertainty in
ecosystem models: The paradox of use for living marine resource
management''. In: \emph{Progress in Oceanography} 102, pp.~102-114. DOI:
\href{https://doi.org/10.1016\%2Fj.pocean.2012.03.008}{10.1016/j.pocean.2012.03.008}.

\protect\hyperlink{cite-moksness_bernard_2012}{Moksness, E, J. Link, K.
Drinkwater, and S. Gaichas} (2012).
``Bernard Megrey: pioneer of Comparative Marine Ecosystem analyses''. In:
\emph{Marine Ecology Progress Series} 459, pp.~165-167. DOI:
\href{https://doi.org/10.3354\%2Fmeps09582}{10.3354/meps09582}.

\protect\hyperlink{cite-pranovi_trophic-level_2012}{Pranovi, F, J. Link, C. Fu,
A. Cook, H. Liu, S. Gaichas, K. Friedland, K. Rong Utne, and H.
Benoît} (2012). ``Trophic-level
determinants of biomass accumulation in marine ecosystems''. In: \emph{Marine
Ecology Progress Series} 459, pp.~185-201. DOI:
\href{https://doi.org/10.3354\%2Fmeps09738}{10.3354/meps09738}.

\protect\hyperlink{cite-gaichas_what_2011}{Gaichas, S. K, K. Y. Aydin, and R.
C. Francis} (2011). ``What drives dynamics in
the Gulf of Alaska? Integrating hypotheses of species, fishing, and
climate relationships using ecosystem modeling''. In: \emph{Canadian Journal
of Fisheries and Aquatic Sciences} 68, pp.~1553-1578.

\protect\hyperlink{cite-hunsicker_functional_2011}{Hunsicker, M. E, L.
Ciannelli, K. M. Bailey, J. A. Buckel, J. W. White, J. S. Link, T. E.
Essington, S. Gaichas, T. W. Anderson, R. D. Brodeur, K. Chan, K. Chen,
G. Englund, K. T. Frank, V. Freitas, M. A. Hixon, T. Hurst, D. W.
Johnson, J. F. Kitchell, D. Reese, G. A. Rose, H. Sjodin, W. J.
Sydeman, H. W. v. d.~Veer, K. Vollset, and S.
Zador} (2011). ``Functional responses
and scaling in predator--prey interactions of marine fishes:
contemporary issues and emerging concepts''. In: \emph{Ecology Letters}
14.12, pp.~1288-1299. DOI:
\href{https://doi.org/10.1111\%2Fj.1461-0248.2011.01696.x\%4010.1111\%2F\%28ISSN\%291461-0248.anthropogenic-change}{10.1111/j.1461-0248.2011.01696.x@10.1111/(ISSN)1461-0248.anthropogenic-change}.

\protect\hyperlink{cite-salomon_bridging_2011}{Salomon, A. K, S. K. Gaichas, O.
P. Jensen, V. N. Agostini, N. Sloan, J. Rice, T. R. McClanahan, M. H.
Ruckelshaus, P. S. Levin, N. K. Dulvy, and E. A.
Babcock} (2011). ``Bridging the Divide
Between Fisheries and Marine Conservation Science''. In: \emph{Bulletin of
Marine Science} 87.2, pp.~251-274. DOI:
\href{https://doi.org/10.5343\%2Fbms.2010.1089}{10.5343/bms.2010.1089}.

\protect\hyperlink{cite-gaichas_using_2010}{Gaichas, S. K, K. Y. Aydin, and R.
C. Francis} (2010). ``Using food web model
results to inform stock assessment estimates of mortality and
production for ecosystem-based fisheries management''. In: \emph{Canadian
Journal of Fisheries and Aquatic Sciences} 67, pp.~1493-1506.

\protect\hyperlink{cite-reuter_managing_2010}{Reuter, R. F, M. E. Conners, J.
Dicosimo, S. Gaichas, O. Ormseth, and T. T.
Tenbrink} (2010). ``Managing non-target,
data-poor species using catch limits: lessons from the Alaskan
groundfish fishery''. In: \emph{Fisheries Management and Ecology} 17.4, pp.
323-335. DOI:
\href{https://doi.org/10.1111\%2Fj.1365-2400.2009.00726.x}{10.1111/j.1365-2400.2009.00726.x}.

\protect\hyperlink{cite-salomon_key_2010}{Salomon, A. K, S. K. Gaichas, N. T.
Shears, J. E. Smith, E. M. P. Madin, and S. D.
Gaines} (2010). ``Key Features and
Context-Dependence of Fishery-Induced Trophic Cascades''. In:
\emph{Conservation Biology} 24.2, pp.~382-394. DOI:
\href{https://doi.org/10.1111\%2Fj.1523-1739.2009.01436.x}{10.1111/j.1523-1739.2009.01436.x}.

\protect\hyperlink{cite-gaichas_comparison_2009}{Gaichas, S, G. Skaret, J.
Falk-Petersen, J. S. Link, W. Overholtz, B. A. Megrey, H. Gjøsæter, W.
T. Stockhausen, A. Dommasnes, K. D. Friedland, and K.
Aydin} (2009). ``A comparison of
community and trophic structure in five marine ecosystems based on
energy budgets and system metrics''. In: \emph{Progress in Oceanography}
81.1-4, pp.~47-62. DOI:
\href{https://doi.org/10.1016\%2Fj.pocean.2009.04.005}{10.1016/j.pocean.2009.04.005}.

\protect\hyperlink{cite-link_comparison_2009}{Link, J. S, W. T. Stockhausen, G.
Skaret, W. Overholtz, B. A. Megrey, H. Gjøsæter, S. Gaichas, A.
Dommasnes, J. Falk-Petersen, J. Kane, F. J. Mueter, K. D. Friedland,
and J. A. Hare} (2009). ``A comparison of
biological trends from four marine ecosystems: Synchronies,
differences, and commonalities''. In: \emph{Progress in Oceanography}.
Comparative Marine Ecosystem Structure and Function: Descriptors and
Characteristics 81.1, pp.~29-46. DOI:
\href{https://doi.org/10.1016\%2Fj.pocean.2009.04.004}{10.1016/j.pocean.2009.04.004}.

\protect\hyperlink{cite-megrey_cross-ecosystem_2009}{Megrey, B. A, J. A. Hare,
W. T. Stockhausen, A. Dommasnes, H. Gjøsæter, W. Overholtz, S. Gaichas,
G. Skaret, J. Falk-Petersen, J. S. Link, and K. D.
Friedland} (2009). ``A
cross-ecosystem comparison of spatial and temporal patterns of
covariation in the recruitment of functionally analogous fish stocks''.
In: \emph{Progress in Oceanography}. Comparative Marine Ecosystem Structure
and Function: Descriptors and Characteristics 81.1, pp.~63-92. DOI:
\href{https://doi.org/10.1016\%2Fj.pocean.2009.04.006}{10.1016/j.pocean.2009.04.006}.

\protect\hyperlink{cite-gaichas_context_2008}{Gaichas,
S.} (2008). ``A context for ecosystem-based
fishery management: Developing concepts of ecosystems and
sustainability''. In: \emph{Marine Policy} 32, pp.~393-401.

\protect\hyperlink{cite-gaichas_network_2008}{Gaichas, S. K. and R. C.
Francis} (2008). ``Network models for
ecosystem-based fishery analysis: A review of concepts and application
to the Gulf of Alaska marine food web''. In: \emph{Canadian Journal of
Fisheries and Aquatic Sciences} 65, pp.~1965-1982.

\protect\hyperlink{cite-aydin_comparison_2007}{Aydin, K. Y., S. Gaichas, I.
Ortiz, D. Kinzey, and N. Friday} (2007).
\emph{A comparison of the Bering Sea, Gulf of Alaska, and Aleutian Islands
large marine ecosystems through food web modeling}. U.S. Department of
Commerce, NOAA Tech. Memo. NMFS-AFSC-178, p.~298.

\protect\hyperlink{cite-gburski_age_2007}{Gburski, C. M, S. K. Gaichas, and D.
K. Kimura} (2007). ``Age and growth of big skate
(Raja binoculata) and longnose skate (R. rhina) in the Gulf of Alaska''.
In: \emph{Environmental Biology of Fishes} 80.2-3, pp.~337-349. DOI:
\href{https://doi.org/10.1007\%2Fs10641-007-9231-8}{10.1007/s10641-007-9231-8}.

\protect\hyperlink{cite-gaichas_development_2006}{Gaichas,
S.} (2006). ``Development and
application of ecosystem models to support fishery sustainability: A
case study for the Gulf of Alaska - ProQuest''. Seattle, WA.

\protect\hyperlink{cite-spies_dna-based_2006}{Spies, I. B, S. Gaichas, D. E.
Stevenson, J. W. Orr, and M. F. Canino}
(2006). ``DNA-based identification of Alaska skates (Amblyraja,
Bathyraja and Raja: Rajidae) using cytochrome c oxidase subunit I (coI)
variation''. In: \emph{Journal of Fish Biology} 69.sb, pp.~283-292. DOI:
\href{https://doi.org/10.1111\%2Fj.1095-8649.2006.01286.x}{10.1111/j.1095-8649.2006.01286.x}.

\protect\hyperlink{cite-barbeaux_evaluation_2005}{Barbeaux, S. J, S. Gaichas,
J. N. Ianelli, and M. W. Dorn} (2005).
``Evaluation of Biological Sampling Protocols for At-Sea Groundfish
Observers in Alaska''. In: \emph{Alaska Fishery Research Bulletin} 11.2, p.
25.

\protect\hyperlink{cite-karp_government-industry_2001}{Karp, W. A, C. S. Rose,
J. R. Gauvin, S. K. Gaichas, M. W. Dorn, and G. D.
Stauffer} (2001).
``Government-Industry Cooperative Fisheries Research in the North
Pacific under the MSFCMA''. In: \emph{Marine Fisheries Review} 63.1, pp.
40-46.

\protect\hyperlink{cite-dorn_measuring_1999}{Dorn, M. W, S. K. Gaichas, S. M.
Fitzgerald, and S. A. Bibb} (1999).
``Measuring Total Catch at Sea: Use of a Motion-Compensated Flow Scale
to Evaluate Observer Volumetric Methods''. In: \emph{North American Journal
of Fisheries Management} 19.4, pp.~999-1016. DOI:
\href{https://doi.org/10.1577\%2F1548-8675\%281999\%29019\%3C0999\%3AMTCASU\%3E2.0.CO\%3B2}{10.1577/1548-8675(1999)019\textless0999:MTCASU\textgreater2.0.CO;2}.

\protect\hyperlink{cite-gaichas_age_1997}{Gaichas, S.}
(1997). ``Age And Growth Of Spanish Mackerel, Scomberomorus Maculatus,
In The Chesapeake Bay Region''. In: \emph{Master's Thesis}. DOI:
\href{https://doi.org/10.25773\%2FV5-TWBW-6T04}{10.25773/V5-TWBW-6T04}.

\hypertarget{scientific-presentations}{%
\section{Scientific Presentations}\label{scientific-presentations}}

Speaker, moderator, and symposium organizer at regional to international professional meetings (AFS, ASLO, Conservation Biology, ESA, GLOBEC, IMCC, World Fisheries Congress) 1995-present. Invited speaker for public lecture series, teacher workshops, and university courses and departmental seminars.

\end{document}
