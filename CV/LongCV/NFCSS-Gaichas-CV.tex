%!TEX TS-program = xelatex
%!TEX encoding = UTF-8 Unicode
% Awesome CV LaTeX Template for CV/Resume
%
% This template has been downloaded from:
% https://github.com/posquit0/Awesome-CV
%
% Author:
% Claud D. Park <posquit0.bj@gmail.com>
% http://www.posquit0.com
%
%
% Adapted to be an Rmarkdown template by Mitchell O'Hara-Wild
% 23 November 2018
%
% Template license:
% CC BY-SA 4.0 (https://creativecommons.org/licenses/by-sa/4.0/)
%
%-------------------------------------------------------------------------------
% CONFIGURATIONS
%-------------------------------------------------------------------------------
% A4 paper size by default, use 'letterpaper' for US letter
\documentclass[11pt, a4paper]{awesome-cv}

% Configure page margins with geometry
\geometry{left=1.4cm, top=.8cm, right=1.4cm, bottom=1.8cm, footskip=.5cm}

% Specify the location of the included fonts
\fontdir[fonts/]

% Color for highlights
% Awesome Colors: awesome-emerald, awesome-skyblue, awesome-red, awesome-pink, awesome-orange
%                 awesome-nephritis, awesome-concrete, awesome-darknight

\colorlet{awesome}{awesome-red}

% Colors for text
% Uncomment if you would like to specify your own color
% \definecolor{darktext}{HTML}{414141}
% \definecolor{text}{HTML}{333333}
% \definecolor{graytext}{HTML}{5D5D5D}
% \definecolor{lighttext}{HTML}{999999}

% Set false if you don't want to highlight section with awesome color
\setbool{acvSectionColorHighlight}{true}

% If you would like to change the social information separator from a pipe (|) to something else
\renewcommand{\acvHeaderSocialSep}{\quad\textbar\quad}

\def\endfirstpage{\newpage}

%-------------------------------------------------------------------------------
%	PERSONAL INFORMATION
%	Comment any of the lines below if they are not required
%-------------------------------------------------------------------------------
% Available options: circle|rectangle,edge/noedge,left/right

\name{Sarah}{Gaichas}

\position{Research Fishery Biologist ZP-IV}
\address{NOAA NMFS Northeast Fisheries Science Center 166 Water Street,
Woods Hole, MA 02543}

\mobile{+1 508 495 2016}
\email{\href{mailto:Sarah.Gaichas@noaa.gov}{\nolinkurl{Sarah.Gaichas@noaa.gov}}}
\github{sgaichas}

% \gitlab{gitlab-id}
% \stackoverflow{SO-id}{SO-name}
% \skype{skype-id}
% \reddit{reddit-id}


\usepackage{booktabs}

\providecommand{\tightlist}{%
	\setlength{\itemsep}{0pt}\setlength{\parskip}{0pt}}

%------------------------------------------------------------------------------



% Pandoc CSL macros
\newlength{\cslhangindent}
\setlength{\cslhangindent}{1.5em}
\newlength{\csllabelwidth}
\setlength{\csllabelwidth}{3em}
\newenvironment{CSLReferences}[3] % #1 hanging-ident, #2 entry spacing
 {% don't indent paragraphs
  \setlength{\parindent}{0pt}
  % turn on hanging indent if param 1 is 1
  \ifodd #1 \everypar{\setlength{\hangindent}{\cslhangindent}}\ignorespaces\fi
  % set entry spacing
  \ifnum #2 > 0
  \setlength{\parskip}{#2\baselineskip}
  \fi
 }%
 {}
\usepackage{calc}
\newcommand{\CSLBlock}[1]{#1\hfill\break}
\newcommand{\CSLLeftMargin}[1]{\parbox[t]{\csllabelwidth}{#1}}
\newcommand{\CSLRightInline}[1]{\parbox[t]{\linewidth - \csllabelwidth}{#1}}
\newcommand{\CSLIndent}[1]{\hspace{\cslhangindent}#1}

\begin{document}

% Print the header with above personal informations
% Give optional argument to change alignment(C: center, L: left, R: right)
\makecvheader

% Print the footer with 3 arguments(<left>, <center>, <right>)
% Leave any of these blank if they are not needed
% 2019-02-14 Chris Umphlett - add flexibility to the document name in footer, rather than have it be static Curriculum Vitae
\makecvfooter
  {June 2021}
    {Sarah Gaichas~~~·~~~Curriculum Vitae}
  {\thepage}


%-------------------------------------------------------------------------------
%	CV/RESUME CONTENT
%	Each section is imported separately, open each file in turn to modify content
%------------------------------------------------------------------------------



\hypertarget{educational-history}{%
\section{Educational History}\label{educational-history}}

\begin{cventries}
    \cventry{PhD}{University of Washington}{Seattle,WA}{2000 - 2006}{}\vspace{-4.0mm}
    \cventry{M.S.}{College of William and Mary, School of Marine Science}{Gloucester Point,VA}{1994 - 1997}{}\vspace{-4.0mm}
    \cventry{B.A.}{Swarthmore College}{Swarthmore,PA}{1986 - 1991}{}\vspace{-4.0mm}
\end{cventries}

\hypertarget{professional-history}{%
\section{Professional History}\label{professional-history}}

\textbf{Research Fishery Biologist} \emph{NOAA/NMFS Northeast Fisheries
Sci. Center, Woods Hole, MA}

\begin{verbatim}
Ecosystem Assessment, MSE       September 2011 - present
\end{verbatim}

Contribute to integrated ecosystem assessments and management strategy
evaluations (MSEs) for the Northeast U.S. continental shelf by analyzing
research survey, fishery, oceanographic, and economic data, and by
developing and testing multispecies assessment models and ecosystem
indicators. Conduct research on ecosystem based management reference
points balancing multiple objectives. Chair National and Northeast MSE
working groups to build capacity and coordinate work plans. Lead editor,
State of the Ecosystem reports. Provide results in peer-reviewed
literature and to the New England and Mid-Atlantic Fishery Management
Councils.

\begin{verbatim}
Acting Branch Chief     January-April 2017  
\end{verbatim}

Managed Ecosystem Dynamics and Assessment Branch (EDAB) administrative
duties during temporary rotation (simultaneous with regular duties). Led
program meetings, responded to leadership requests, conducted employee
mid-term reviews, coordinated with other Branch Chiefs within the
Division, temporarily acted as Division Chief. Completed 2021 Labor
planning for EDAB (developed plans with branch, participant in
cross-center working group) and developed budget (SPERS) plans for EDAB.

\textbf{Research Fishery Biologist} \emph{NOAA/NMFS Alaska Fisheries
Science Center, Seattle, WA}

\begin{verbatim}
Ecosystem Modeling          June 2006 – September 2011
\end{verbatim}

Established an ecosystem context for fishery management through analysis
of food web, fishery, and research survey data combined with ecosystem
modeling. Evaluated the cumulative effects of fishing, climate change,
and other factors on the marine ecosystem and species within it.
Conducted independent research on ecosystem level indicators and
thresholds and provided results in peer-reviewed literature and to the
North Pacific Fishery Management Council. Participated in research
surveys at sea.

\begin{verbatim}
Stock Assessment            November 1998 – June 2006
\end{verbatim}

Assessed the condition of commercially exploited fish stocks through
analysis of fishery and research survey data combined with population
dynamics modeling. Evaluated the effects of fishing on non-target
species and the marine ecosystem. Conducted Ph.D.~research. Attended
2001 Complex Systems Summer School, Santa Fe Institute, Santa Fe NM.
Participated in research surveys at sea.

Stock assessments: 1999 Gulf of Alaska (GOA) and 2000 -- 2004 Bering Sea
Aleutian Islands (BSAI) ``Other species'' (all species of skates,
sharks, sculpins, octopi, and squids), 2000-2005 BSAI squids, 2005 BSAI
skates, 2003-2005 GOA skates, 2000-2005 GOA Thornyheads, and 2006 GOA
squids.

\begin{verbatim}
Observer Program            March 1997 – October 1999
\end{verbatim}

Conducted statistical analyses of fishery observer data, evaluated
present and proposed data collection methods through directed research,
and examined uses of observer data in multiple applications.

\hypertarget{awards}{%
\section{Awards}\label{awards}}

\begin{itemize}
\tightlist
\item
  End-year performance bonuses each year, 2006-2020.
\item
  Individual cash award for State of the Ecosystem reporting, August
  2020.
\item
  Individual cash award for Acting Branch Chief performance, July 2017.
\item
  ICES/MYFISH symposium Targets and Limits for Long Term Fisheries
  Management Most Inspiring Paper award, October 2015.
\item
  Department of Commerce ``Cash-In-A-Flash'' award, December 2011.
\item
  Department of Commerce Special Act Award: Alaska Fisheries Science
  Center Employee of the Year, 2010.
\item
  Department of Commerce Special Act Award for service on North Pacific
  Council Gulf of Alaska Plan Team and for ecosystem research
  contributions, July 2005.
\item
  Department of Commerce Special Act Award for innovations in nontarget
  species management and ecosystem modeling, 2004.
\item
  National Association of Environmental Professionals Award for
  Environmental Excellence in NEPA for assistance with the Steller Seal
  Lion Protection Measures SEIS, June 2003.
\item
  Mote International Symposium on Confronting Trade-offs in the
  Ecosystem Approach to Fisheries Management Young Investigator Award,
  November 2002.
\item
  Department of Commerce Special Act Award for service as the Nontarget
  and Prohibited Species Team Leader in the development of the Draft
  Programmatic SEIS for groundfish fisheries in Alaska, September 2000.
\item
  Department of Commerce ``Cash-In-A-Flash'' awards, March and June
  1998.
\item
  College of William and Mary William J. Hargis Jr.~Fellowship Award and
  Dean's Fellowship. April, 1995.
\end{itemize}

\hypertarget{publications}{%
\section{Publications}\label{publications}}

\hypertarget{papers-published-in-peer-reviewed-journals}{%
\subsection{Papers published in peer-reviewed
journals}\label{papers-published-in-peer-reviewed-journals}}

\protect\hyperlink{cite-bastille_improving_2021}{Bastille, K., S.
Hardison, L. deWitt, J. Brown, J. Samhouri, S. Gaichas, S. Lucey, K.
Kearney, B. Best, S. Cross, S. Large, and E. Spooner} (2021).
``Improving the IEA Approach Using Principles of Open Data Science.''
In: \emph{Coastal Management} 49.1, pp.~72-89. DOI:
\href{https://doi.org/10.1080\%2F08920753.2021.1846155}{10.1080/08920753.2021.1846155}.

\protect\hyperlink{cite-depiper_learning_2021}{DePiper, G., S. Gaichas,
B. Muffley, G. Ardini, J. Brust, J. Coakley, K. Dancy, G. W. Elliott, D.
C. Leaning, D. Lipton, J. McNamee, C. Perretti, K. Rootes-Murdy, and M.
J. Wilberg} (2021). ``Learning by doing: collaborative conceptual
modelling as a path forward in ecosystem-based management.'' In:
\emph{ICES Journal of Marine Science}. DOI:
\href{https://doi.org/10.1093\%2Ficesjms\%2Ffsab054}{10.1093/icesjms/fsab054}.

\protect\hyperlink{cite-friedland_resource_2021}{Friedland, K. D., E. T.
Methratta, A. B. Gill, S. K. Gaichas, T. H. Curtis, E. M. Adams, J. L.
Morano, D. P. Crear, M. C. McManus, and D. C. Brady} (2021). ``Resource
Occurrence and Productivity in Existing and Proposed Wind Energy Lease
Areas on the Northeast US Shelf.'' In: \emph{Frontiers in Marine
Science} 8. DOI:
\href{https://doi.org/10.3389\%2Ffmars.2021.629230}{10.3389/fmars.2021.629230}.

\protect\hyperlink{cite-lucey_evaluating_2021}{Lucey, S. M., K. Y.
Aydin, S. K. Gaichas, S. X. Cadrin, G. Fay, M. J. Fogarty, and A. Punt}
(2021). ``Evaluating fishery management strategies using an ecosystem
model as an operating model.'' In: \emph{Fisheries Research} 234,
p.~105780. DOI:
\href{https://doi.org/10.1016\%2Fj.fishres.2020.105780}{10.1016/j.fishres.2020.105780}.

\protect\hyperlink{cite-muffley_there_2021}{Muffley, B., S. Gaichas, G.
DePiper, R. Seagraves, and S. Lucey} (2021). ``There Is no I in EAFM
Adapting Integrated Ecosystem Assessment for Mid-Atlantic Fisheries
Management.'' In: \emph{Coastal Management} 49.1, pp. 90-106. DOI:
\href{https://doi.org/10.1080\%2F08920753.2021.1846156}{10.1080/08920753.2021.1846156}.

\protect\hyperlink{cite-reum_its_2021}{Reum, J. C. P., H. Townsend, S.
Gaichas, S. Sagarese, I. C. Kaplan, and A. Grüss} (2021). ``It's Not the
Destination, It's the Journey: Multispecies Model Ensembles for
Ecosystem Approaches to Fisheries Management.'' In: \emph{Frontiers in
Marine Science} 8. DOI:
\href{https://doi.org/10.3389\%2Ffmars.2021.631839}{10.3389/fmars.2021.631839}.

\protect\hyperlink{cite-link_changing_2020}{Link, J. S., G. Huse, S.
Gaichas, and A. R. Marshak} (2020). ``Changing how we approach
fisheries: A first attempt at an operational framework for ecosystem
approaches to fisheries management.'' In: \emph{Fish and Fisheries}
21.2, pp.~393-434. DOI:
\href{https://doi.org/10.1111\%2Ffaf.12438}{10.1111/faf.12438}.

\protect\hyperlink{cite-lucey_conducting_2020}{Lucey, S. M., S. K.
Gaichas, and K. Y. Aydin} (2020). ``Conducting reproducible ecosystem
modeling using the open source mass balance model Rpath.'' In:
\emph{Ecological Modelling} 427, p.~109057. DOI:
\href{https://doi.org/10.1016\%2Fj.ecolmodel.2020.109057}{10.1016/j.ecolmodel.2020.109057}.

\protect\hyperlink{cite-weiskopf_climate_2020}{Weiskopf, S. R., M. A.
Rubenstein, L. G. Crozier, S. Gaichas, R. Griffis, J. E. Halofsky, K. J.
W. Hyde, T. L. Morelli, J. T. Morisette, R. C. Muñoz, A. J. Pershing, D.
L. Peterson, R. Poudel, M. D. Staudinger, A. E. Sutton-Grier, L.
Thompson, J. Vose, J. F. Weltzin, and K. P. Whyte} (2020). ``Climate
change effects on biodiversity, ecosystems, ecosystem services, and
natural resource management in the United States.'' In: \emph{Science of
The Total Environment}, p.~137782. DOI:
\href{https://doi.org/10.1016\%2Fj.scitotenv.2020.137782}{10.1016/j.scitotenv.2020.137782}.

\protect\hyperlink{cite-goethel_recent_2019}{Goethel, D. R., S. M.
Lucey, A. M. Berger, S. K. Gaichas, M. A. Karp, P. D. Lynch, and J. F.
Walter} (2019). ``Recent advances in management strategy evaluation:
introduction to the special issue ``Under pressure: addressing fisheries
challenges with Management Strategy Evaluation''". In: \emph{Canadian
Journal of Fisheries and Aquatic Sciences} 76.10, pp.~1689-1696. DOI:
\href{https://doi.org/10.1139\%2Fcjfas-2019-0084}{10.1139/cjfas-2019-0084}.

\protect\hyperlink{cite-karp_accounting_2019}{Karp, M. A., J. O.
Peterson, P. D. Lynch, R. B. Griffis, C. F. Adams, W. S. Arnold, L. A.
K. Barnett, Y. deReynier, J. DiCosimo, K. H. Fenske, S. K. Gaichas, A.
Hollowed, K. Holsman, M. Karnauskas, D. Kobayashi, A. Leising, J. P.
Manderson, M. McClure, W. E. Morrison, E. Schnettler, A. Thompson, J. T.
Thorson, J. F. Walter, A. J. Yau, R. D. Methot, and J. S. Link} (2019).
``Accounting for shifting distributions and changing productivity in the
development of scientific advice for fishery management.'' In:
\emph{ICES Journal of Marine Science} 76.5, pp.~1305-1315. DOI:
\href{https://doi.org/10.1093\%2Ficesjms\%2Ffsz048}{10.1093/icesjms/fsz048}.

\protect\hyperlink{cite-townsend_progress_2019}{Townsend, H., C. J.
Harvey, Y. deReynier, D. Davis, S. G. Zador, S. Gaichas, M. Weijerman,
E. L. Hazen, and I. C. Kaplan} (2019). ``Progress on Implementing
Ecosystem-Based Fisheries Management in the United States Through the
Use of Ecosystem Models and Analysis.'' In: \emph{Frontiers in Marine
Science} 6. DOI:
\href{https://doi.org/10.3389\%2Ffmars.2019.00641}{10.3389/fmars.2019.00641}.

\protect\hyperlink{cite-deroba_dream_2018}{Deroba, J. J., S. K. Gaichas,
M. Lee, R. G. Feeney, D. V. Boelke, and B. J. Irwin} (2018). ``The dream
and the reality: meeting decision-making time frames while incorporating
ecosystem and economic models into management strategy evaluation.'' In:
\emph{Canadian Journal of Fisheries and Aquatic Sciences}. DOI:
\href{https://doi.org/10.1139\%2Fcjfas-2018-0128}{10.1139/cjfas-2018-0128}.

\protect\hyperlink{cite-feeney_integrating_2018}{Feeney, R. G., D. V.
Boelke, J. J. Deroba, S. K. Gaichas, B. J. Irwin, and M. Lee} (2018).
``Integrating Management Strategy Evaluation into fisheries management:
advancing best practices for stakeholder inclusion based on an MSE for
Northeast U.S. Atlantic herring.'' In: \emph{Canadian Journal of
Fisheries and Aquatic Sciences}. DOI:
\href{https://doi.org/10.1139\%2Fcjfas-2018-0125}{10.1139/cjfas-2018-0125}.

\protect\hyperlink{cite-gaichas_implementing_2018}{Gaichas, S. K., G. S.
DePiper, R. J. Seagraves, B. W. Muffley, M. Sabo, L. L. Colburn, and A.
L. Loftus} (2018). ``Implementing Ecosystem Approaches to Fishery
Management: Risk Assessment in the US Mid-Atlantic.'' In:
\emph{Frontiers in Marine Science} 5. DOI:
\href{https://doi.org/10.3389\%2Ffmars.2018.00442}{10.3389/fmars.2018.00442}.

\protect\hyperlink{cite-goethel_closing_2018}{Goethel, D. R., S. M.
Lucey, A. M. Berger, S. K. Gaichas, M. A. Karp, P. D. Lynch, J. F.
Walter III, J. J. Deroba, S. K. Miller, and M. J. Wilberg} (2018).
``Closing the Feedback Loop: On Stakeholder Participation in Management
Strategy Evaluation.'' In: \emph{Canadian Journal of Fisheries and
Aquatic Sciences}. DOI:
\href{https://doi.org/10.1139\%2Fcjfas-2018-0162}{10.1139/cjfas-2018-0162}.

\protect\hyperlink{cite-lipton_ecosystems_2018}{Lipton, D., M. A.
Rubenstein, S. R. Weiskopf, S. L. Carter, J. Peterson, L. Crozier, M.
Fogarty, S. Gaichas, K. J. W. Hyde, T. L. Morelli, J. Morisette, H.
Moustahfid, R. Munoz, R. Poudel, M. D. Staudinger, C. Stock, L.
Thompson, R. S. Waples, and J. F. Weltzin} (2018). \emph{Ecosystems,
Ecosystem Services, and Biodiversity}. Other Government Series. U.S.
Global Change Research Program, p.~268321. DOI:
\href{https://doi.org/10.7930\%2FNCA4.2018.CH7}{10.7930/NCA4.2018.CH7}.

\protect\hyperlink{cite-olsen_ocean_2018}{Olsen, E., I. C. Kaplan, C.
Ainsworth, G. Fay, S. Gaichas, R. Gamble, R. Girardin, C. H. Eide, T. F.
Ihde, H. N. Morzaria-Luna, K. F. Johnson, M. Savina-Rolland, H.
Townsend, M. Weijerman, E. A. Fulton, and J. S. Link} (2018). ``Ocean
Futures Under Ocean Acidification, Marine Protection, and Changing
Fishing Pressures Explored Using a Worldwide Suite of Ecosystem
Models.'' In: \emph{Frontiers in Marine Science} 5. DOI:
\href{https://doi.org/10.3389\%2Ffmars.2018.00064}{10.3389/fmars.2018.00064}.

\protect\hyperlink{cite-depiper_operationalizing_2017}{DePiper, G. S.,
S. K. Gaichas, S. M. Lucey, P. Pinto da Silva, M. R. Anderson, H.
Breeze, A. Bundy, P. M. Clay, G. Fay, R. J. Gamble, R. S. Gregory, P. S.
Fratantoni, C. L. Johnson, M. Koen-Alonso, K. M. Kleisner, J. Olson, C.
T. Perretti, P. Pepin, F. Phelan, V. S. Saba, L. A. Smith, J. C. Tam, N.
D. Templeman, and R. P. Wildermuth} (2017). ``Operationalizing
integrated ecosystem assessments within a multidisciplinary team:
lessons learned from a worked example.'' In: \emph{ICES Journal of
Marine Science} 74.8, pp.~2076-2086. DOI:
\href{https://doi.org/10.1093\%2Ficesjms\%2Ffsx038}{10.1093/icesjms/fsx038}.

\protect\hyperlink{cite-gaichas_combining_2017}{Gaichas, S. K., M.
Fogarty, G. Fay, R. Gamble, S. Lucey, and L. Smith} (2017). ``Combining
stock, multispecies, and ecosystem level fishery objectives within an
operational management procedure: simulations to start the
conversation.'' In: \emph{ICES Journal of Marine Science} 74.2, pp.
552-565. DOI:
\href{https://doi.org/10.1093\%2Ficesjms\%2Ffsw119}{10.1093/icesjms/fsw119}.

\protect\hyperlink{cite-holsman_ecosystem-based_2017}{Holsman, K., J.
Samhouri, G. Cook, E. Hazen, E. Olsen, M. Dillard, S. Kasperski, S.
Gaichas, C. R. Kelble, M. Fogarty, and K. Andrews} (2017). ``An
ecosystem-based approach to marine risk assessment.'' In:
\emph{Ecosystem Health and Sustainability} 3.1, p.~e01256. DOI:
\href{https://doi.org/10.1002\%2Fehs2.1256}{10.1002/ehs2.1256}.

\protect\hyperlink{cite-tommasi_managing_2017}{Tommasi, D., C. A. Stock,
A. J. Hobday, R. Methot, I. C. Kaplan, J. P. Eveson, K. Holsman, T. J.
Miller, S. Gaichas, M. Gehlen, A. Pershing, G. A. Vecchi, R. Msadek, T.
Delworth, C. M. Eakin, M. A. Haltuch, R. Séférian, C. M. Spillman, J. R.
Hartog, S. Siedlecki, J. F. Samhouri, B. Muhling, R. G. Asch, M. L.
Pinsky, V. S. Saba, S. B. Kapnick, C. F. Gaitan, R. R. Rykaczewski, M.
A. Alexander, Y. Xue, K. V. Pegion, P. Lynch, M. R. Payne, T.
Kristiansen, P. Lehodey, and F. E. Werner} (2017). ``Managing living
marine resources in a dynamic environment: The role of seasonal to
decadal climate forecasts.'' In: \emph{Progress in Oceanography} 152,
pp.~15-49. DOI:
\href{https://doi.org/10.1016\%2Fj.pocean.2016.12.011}{10.1016/j.pocean.2016.12.011}.

\protect\hyperlink{cite-wildermuth_structural_2017}{Wildermuth, R. P.,
G. Fay, and S. Gaichas} (2017). ``Structural uncertainty in qualitative
models for ecosystem-based management of Georges Bank.'' In:
\emph{Canadian Journal of Fisheries and Aquatic Sciences} 75.10,
pp.~1635-1643. DOI:
\href{https://doi.org/10.1139\%2Fcjfas-2017-0149}{10.1139/cjfas-2017-0149}.

\protect\hyperlink{cite-zador_linking_2017}{Zador, S. G., S. K. Gaichas,
S. Kasperski, C. L. Ward, R. E. Blake, N. C. Ban, A. Himes-Cornell, and
J. Z. Koehn} (2017). ``Linking ecosystem processes to communities of
practice through commercially fished species in the Gulf of Alaska.''
In: \emph{ICES Journal of Marine Science} 74.7, pp.~2024-2033. DOI:
\href{https://doi.org/10.1093\%2Ficesjms\%2Ffsx054}{10.1093/icesjms/fsx054}.

\protect\hyperlink{cite-gaichas_framework_2016}{Gaichas, S. K., R. J.
Seagraves, J. M. Coakley, G. S. DePiper, V. G. Guida, J. A. Hare, P. J.
Rago, and M. J. Wilberg} (2016). ``A Framework for Incorporating
Species, Fleet, Habitat, and Climate Interactions into Fishery
Management.'' In: \emph{Frontiers in Marine Science} 3. DOI:
\href{https://doi.org/10.3389\%2Ffmars.2016.00105}{10.3389/fmars.2016.00105}.

\protect\hyperlink{cite-olsen_ecosystem_2016}{Olsen, E., G. Fay, S.
Gaichas, R. Gamble, S. Lucey, and J. S. Link} (2016). ``Ecosystem Model
Skill Assessment. Yes We Can!'' In: \emph{PLOS ONE} 11.1. Ed. by C. N.
Bianchi, p.~e0146467. DOI:
\href{https://doi.org/10.1371\%2Fjournal.pone.0146467}{10.1371/journal.pone.0146467}.

\protect\hyperlink{cite-weijerman_atlantis_2016}{Weijerman, M., J. Link,
E. Fulton, E. Olsen, H. Townsend, S. Gaichas, C. Hansen, M.
Skern-Mauritzen, I. Kaplan, R. Gamble, G. Fay, M. Savina, C. Ainsworth,
I. van Putten, R. Gorton, R. Brainard, K. Larsen, and T. Hutton} (2016).
``Atlantis Ecosystem Model Summit: Report from a workshop.'' In:
\emph{Ecological Modelling} 335, pp.~35-38. DOI:
\href{https://doi.org/10.1016\%2Fj.ecolmodel.2016.05.007}{10.1016/j.ecolmodel.2016.05.007}.

\protect\hyperlink{cite-zador_ecosystem_2016}{Zador, S. G., K. K.
Holsman, K. Y. Aydin, and S. K. Gaichas} (2016). ``Ecosystem
considerations in Alaska: the value of qualitative assessments.'' In:
\emph{ICES Journal of Marine Science: Journal du Conseil}, p.~fsw144.
DOI:
\href{https://doi.org/10.1093\%2Ficesjms\%2Ffsw144}{10.1093/icesjms/fsw144}.

\protect\hyperlink{cite-gaichas_wasp_2015}{Gaichas, S., K. Aydin, and R.
C. Francis} (2015). ``Wasp waist or beer belly? Modeling food web
structure and energetic control in Alaskan marine ecosystems, with
implications for fishing and environmental forcing.'' In: \emph{Progress
in Oceanography} 138, Part A, pp.~1-17. DOI:
\href{https://doi.org/10.1016\%2Fj.pocean.2015.09.010}{10.1016/j.pocean.2015.09.010}.

\protect\hyperlink{cite-smith_simulations_2015}{Smith, L., R. Gamble, S.
Gaichas, and J. Link} (2015). ``Simulations to evaluate management
trade-offs among marine mammal consumption needs, commercial fishing
fleets and finfish biomass.'' In: \emph{Marine Ecology Progress Series}
523, pp.~215-232. DOI:
\href{https://doi.org/10.3354\%2Fmeps11129}{10.3354/meps11129}.

\protect\hyperlink{cite-gaichas_risk-based_2014}{Gaichas, S. K., J. S.
Link, and J. A. Hare} (2014). ``A risk-based approach to evaluating
northeast US fish community vulnerability to climate change.'' In:
\emph{ICES Journal of Marine Science} 71.8, pp. 2323-2342. DOI:
\href{https://doi.org/10.1093\%2Ficesjms\%2Ffsu048}{10.1093/icesjms/fsu048}.

\protect\hyperlink{cite-ruzicka_dividing_2013}{Ruzicka, J. J., J. H.
Steele, T. Ballerini, S. K. Gaichas, and D. G. Ainley} (2013).
``Dividing up the pie: Whales, fish, and humans as competitors.'' In:
\emph{Progress in Oceanography} 116, pp.~207-219. DOI:
\href{https://doi.org/10.1016\%2Fj.pocean.2013.07.009}{10.1016/j.pocean.2013.07.009}.

\protect\hyperlink{cite-ruzicka_analysis_2013}{Ruzicka, J., J. Steele,
S. Gaichas, T. Ballerini, D. Gifford, R. Brodeur, and E. Hofmann}
(2013). ``Analysis of Energy Flow in US GLOBEC Ecosystems Using
End-to-End Models.'' In: \emph{Oceanography} 26.4, pp.~82-97. DOI:
\href{https://doi.org/10.5670\%2Foceanog.2013.77}{10.5670/oceanog.2013.77}.

\protect\hyperlink{cite-fu_relative_2012}{Fu, C., S. Gaichas, J. Link,
A. Bundy, J. Boldt, A. Cook, R. Gamble, K. Rong Utne, H. Liu, and K.
Friedland} (2012). ``Relative importance of fisheries, trophodynamic and
environmental drivers in a series of marine ecosystems.'' In:
\emph{Marine Ecology Progress Series} 459, pp. 169-184. DOI:
\href{https://doi.org/10.3354\%2Fmeps09805}{10.3354/meps09805}.

\protect\hyperlink{cite-gaichas_beyond_2012}{Gaichas, S. K., G. Odell,
K. Y. Aydin, and R. C. Francis} (2012). ``Beyond the defaults:
functional response parameter space and ecosystem-level fishing
thresholds in dynamic food web model simulations.'' In: \emph{Canadian
Journal of Fisheries and Aquatic Sciences} 69, pp.~2077-2094. DOI:
\href{https://doi.org/10.1139\%2FF2012-099}{10.1139/F2012-099}.

\protect\hyperlink{cite-gaichas_what_2012}{Gaichas, S., A. Bundy, T.
Miller, E. Moksness, and K. Stergiou} (2012). ``What drives marine
fisheries production?'' In: \emph{Marine Ecology Progress Series} 459,
pp.~159-163. DOI:
\href{https://doi.org/10.3354\%2Fmeps09841}{10.3354/meps09841}.

\protect\hyperlink{cite-gaichas_assembly_2012}{Gaichas, S., R. Gamble,
M. J. Fogarty, H. Benoit, T. Essington, C. Fu, M. Koen-Alonso, and J.
Link} (2012). ``Assembly Rules for Aggregate-Species Production Models:
Simulations in Support of Management Strategy Evaluation.'' In:
\emph{Marine Ecology Progress Series} 459, pp.~275-292.

\protect\hyperlink{cite-link_synthesizing_2012}{Link, J., S. Gaichas, T.
Miller, T. Essington, A. Bundy, J. Boldt, K. Drinkwater, and E.
Moksness} (2012). ``Synthesizing Lessons Learned from Comparing
Fisheries Production in 13 Northern Hemisphere Ecosystems: Emergent
Fundamental Features.'' In: \emph{Marine Ecology Progress Series} 459,
pp.~293-302.

\protect\hyperlink{cite-link_dealing_2012}{Link, J., T. Ihde, C. Harvey,
S. Gaichas, J. Field, J. Brodziak, H. Townsend, and R. Peterman} (2012).
``Dealing with uncertainty in ecosystem models: The paradox of use for
living marine resource management.'' In: \emph{Progress in Oceanography}
102, pp.~102-114. DOI:
\href{https://doi.org/10.1016\%2Fj.pocean.2012.03.008}{10.1016/j.pocean.2012.03.008}.

\protect\hyperlink{cite-moksness_bernard_2012}{Moksness, E., J. Link, K.
Drinkwater, and S. Gaichas} (2012). ``Bernard Megrey: pioneer of
Comparative Marine Ecosystem analyses.'' In: \emph{Marine Ecology
Progress Series} 459, pp.~165-167. DOI:
\href{https://doi.org/10.3354\%2Fmeps09582}{10.3354/meps09582}.

\protect\hyperlink{cite-pranovi_trophic-level_2012}{Pranovi, F., J.
Link, C. Fu, A. Cook, H. Liu, S. Gaichas, K. Friedland, K. Rong Utne,
and H. Benoît} (2012). ``Trophic-level determinants of biomass
accumulation in marine ecosystems.'' In: \emph{Marine Ecology Progress
Series} 459, pp.~185-201. DOI:
\href{https://doi.org/10.3354\%2Fmeps09738}{10.3354/meps09738}.

\protect\hyperlink{cite-gaichas_what_2011}{Gaichas, S. K., K. Y. Aydin,
and R. C. Francis} (2011). ``What drives dynamics in the Gulf of Alaska?
Integrating hypotheses of species, fishing, and climate relationships
using ecosystem modeling.'' In: \emph{Canadian Journal of Fisheries and
Aquatic Sciences} 68, pp.~1553-1578.

\protect\hyperlink{cite-hunsicker_functional_2011}{Hunsicker, M. E., L.
Ciannelli, K. M. Bailey, J. A. Buckel, J. W. White, J. S. Link, T. E.
Essington, S. Gaichas, T. W. Anderson, R. D. Brodeur, K. Chan, K. Chen,
G. Englund, K. T. Frank, V. Freitas, M. A. Hixon, T. Hurst, D. W.
Johnson, J. F. Kitchell, D. Reese, G. A. Rose, H. Sjodin, W. J. Sydeman,
H. W. v. d.~Veer, K. Vollset, and S. Zador} (2011). ``Functional
responses and scaling in predator--prey interactions of marine fishes:
contemporary issues and emerging concepts.'' In: \emph{Ecology Letters}
14.12, pp.~1288-1299. DOI:
\href{https://doi.org/10.1111\%2Fj.1461-0248.2011.01696.x\%4010.1111\%2F\%28ISSN\%291461-0248.anthropogenic-change}{10.1111/j.1461-0248.2011.01696.x@10.1111/(ISSN)1461-0248.anthropogenic-change}.

\protect\hyperlink{cite-salomon_bridging_2011}{Salomon, A. K., S. K.
Gaichas, O. P. Jensen, V. N. Agostini, N. Sloan, J. Rice, T. R.
McClanahan, M. H. Ruckelshaus, P. S. Levin, N. K. Dulvy, and E. A.
Babcock} (2011). ``Bridging the Divide Between Fisheries and Marine
Conservation Science.'' In: \emph{Bulletin of Marine Science} 87.2,
pp.~251-274. DOI:
\href{https://doi.org/10.5343\%2Fbms.2010.1089}{10.5343/bms.2010.1089}.

\protect\hyperlink{cite-gaichas_using_2010}{Gaichas, S. K., K. Y. Aydin,
and R. C. Francis} (2010). ``Using food web model results to inform
stock assessment estimates of mortality and production for
ecosystem-based fisheries management.'' In: \emph{Canadian Journal of
Fisheries and Aquatic Sciences} 67, pp.~1493-1506.

\protect\hyperlink{cite-reuter_managing_2010}{Reuter, R. F., M. E.
Conners, J. Dicosimo, S. Gaichas, O. Ormseth, and T. T. Tenbrink}
(2010). ``Managing non-target, data-poor species using catch limits:
lessons from the Alaskan groundfish fishery.'' In: \emph{Fisheries
Management and Ecology} 17.4, pp. 323-335. DOI:
\href{https://doi.org/10.1111\%2Fj.1365-2400.2009.00726.x}{10.1111/j.1365-2400.2009.00726.x}.

\protect\hyperlink{cite-salomon_key_2010}{Salomon, A. K., S. K. Gaichas,
N. T. Shears, J. E. Smith, E. M. P. Madin, and S. D. Gaines} (2010).
``Key Features and Context-Dependence of Fishery-Induced Trophic
Cascades.'' In: \emph{Conservation Biology} 24.2, pp.~382-394. DOI:
\href{https://doi.org/10.1111\%2Fj.1523-1739.2009.01436.x}{10.1111/j.1523-1739.2009.01436.x}.

\protect\hyperlink{cite-gaichas_comparison_2009}{Gaichas, S., G. Skaret,
J. Falk-Petersen, J. S. Link, W. Overholtz, B. A. Megrey, H. Gjøsæter,
W. T. Stockhausen, A. Dommasnes, K. D. Friedland, and K. Aydin} (2009).
``A comparison of community and trophic structure in five marine
ecosystems based on energy budgets and system metrics.'' In:
\emph{Progress in Oceanography} 81.1-4, pp.~47-62. DOI:
\href{https://doi.org/10.1016\%2Fj.pocean.2009.04.005}{10.1016/j.pocean.2009.04.005}.

\protect\hyperlink{cite-link_comparison_2009}{Link, J. S., W. T.
Stockhausen, G. Skaret, W. Overholtz, B. A. Megrey, H. Gjøsæter, S.
Gaichas, A. Dommasnes, J. Falk-Petersen, J. Kane, F. J. Mueter, K. D.
Friedland, and J. A. Hare} (2009). ``A comparison of biological trends
from four marine ecosystems: Synchronies, differences, and
commonalities.'' In: \emph{Progress in Oceanography}. Comparative Marine
Ecosystem Structure and Function: Descriptors and Characteristics 81.1,
pp.~29-46. DOI:
\href{https://doi.org/10.1016\%2Fj.pocean.2009.04.004}{10.1016/j.pocean.2009.04.004}.

\protect\hyperlink{cite-megrey_cross-ecosystem_2009}{Megrey, B. A., J.
A. Hare, W. T. Stockhausen, A. Dommasnes, H. Gjøsæter, W. Overholtz, S.
Gaichas, G. Skaret, J. Falk-Petersen, J. S. Link, and K. D. Friedland}
(2009). ``A cross-ecosystem comparison of spatial and temporal patterns
of covariation in the recruitment of functionally analogous fish
stocks.'' In: \emph{Progress in Oceanography}. Comparative Marine
Ecosystem Structure and Function: Descriptors and Characteristics 81.1,
pp.~63-92. DOI:
\href{https://doi.org/10.1016\%2Fj.pocean.2009.04.006}{10.1016/j.pocean.2009.04.006}.

\protect\hyperlink{cite-gaichas_context_2008}{Gaichas, S.} (2008). ``A
context for ecosystem-based fishery management: Developing concepts of
ecosystems and sustainability.'' In: \emph{Marine Policy} 32,
pp.~393-401.

\protect\hyperlink{cite-gaichas_network_2008}{Gaichas, S. K. and R. C.
Francis} (2008). ``Network models for ecosystem-based fishery analysis:
A review of concepts and application to the Gulf of Alaska marine food
web.'' In: \emph{Canadian Journal of Fisheries and Aquatic Sciences} 65,
pp.~1965-1982.

\protect\hyperlink{cite-aydin_comparison_2007}{Aydin, K. Y., S. Gaichas,
I. Ortiz, D. Kinzey, and N. Friday} (2007). \emph{A comparison of the
Bering Sea, Gulf of Alaska, and Aleutian Islands large marine ecosystems
through food web modeling}. U.S. Department of Commerce, NOAA Tech.
Memo. NMFS-AFSC-178, p.~298.

\protect\hyperlink{cite-gburski_age_2007}{Gburski, C. M., S. K. Gaichas,
and D. K. Kimura} (2007). ``Age and growth of big skate (Raja
binoculata) and longnose skate (R. rhina) in the Gulf of Alaska.'' In:
\emph{Environmental Biology of Fishes} 80.2-3, pp.~337-349. DOI:
\href{https://doi.org/10.1007\%2Fs10641-007-9231-8}{10.1007/s10641-007-9231-8}.

\protect\hyperlink{cite-gaichas_development_2006}{Gaichas, S.} (2006).
``Development and application of ecosystem models to support fishery
sustainability: A case study for the Gulf of Alaska - ProQuest.''
Seattle, WA.

\protect\hyperlink{cite-spies_dna-based_2006}{Spies, I. B., S. Gaichas,
D. E. Stevenson, J. W. Orr, and M. F. Canino} (2006). ``DNA-based
identification of Alaska skates (Amblyraja, Bathyraja and Raja: Rajidae)
using cytochrome c oxidase subunit I (coI) variation.'' In:
\emph{Journal of Fish Biology} 69.sb, pp.~283-292. DOI:
\href{https://doi.org/10.1111\%2Fj.1095-8649.2006.01286.x}{10.1111/j.1095-8649.2006.01286.x}.

\protect\hyperlink{cite-barbeaux_evaluation_2005}{Barbeaux, S. J., S.
Gaichas, J. N. Ianelli, and M. W. Dorn} (2005). ``Evaluation of
Biological Sampling Protocols for At-Sea Groundfish Observers in
Alaska.'' In: \emph{Alaska Fishery Research Bulletin} 11.2, p. 25.

\protect\hyperlink{cite-karp_government-industry_2001}{Karp, W. A., C.
S. Rose, J. R. Gauvin, S. K. Gaichas, M. W. Dorn, and G. D. Stauffer}
(2001). ``Government-Industry Cooperative Fisheries Research in the
North Pacific under the MSFCMA.'' In: \emph{Marine Fisheries Review}
63.1, pp. 40-46.

\protect\hyperlink{cite-dorn_measuring_1999}{Dorn, M. W., S. K. Gaichas,
S. M. Fitzgerald, and S. A. Bibb} (1999). ``Measuring Total Catch at
Sea: Use of a Motion-Compensated Flow Scale to Evaluate Observer
Volumetric Methods.'' In: \emph{North American Journal of Fisheries
Management} 19.4, pp.~999-1016. DOI:
\href{https://doi.org/10.1577\%2F1548-8675\%281999\%29019\%3C0999\%3AMTCASU\%3E2.0.CO\%3B2}{10.1577/1548-8675(1999)019\textless0999:MTCASU\textgreater2.0.CO;2}.

\protect\hyperlink{cite-gaichas_age_1997}{Gaichas, S.} (1997). ``Age And
Growth Of Spanish Mackerel, Scomberomorus Maculatus, In The Chesapeake
Bay Region.'' In: \emph{Master's Thesis}. DOI:
\href{https://doi.org/10.25773\%2FV5-TWBW-6T04}{10.25773/V5-TWBW-6T04}.

\hypertarget{papers-submitted-to-peer-reviewed-journals}{%
\subsection{Papers submitted to peer-reviewed
journals}\label{papers-submitted-to-peer-reviewed-journals}}

Kaplan, I.C., S.K. Gaichas, C.C. Stawitz, P.D. Lynch, K.N. Marshall,
J.J. Deroba, M. Masi, J.K.T. Brodziak, K.Y. Aydin, K. Holsman, H.
Townsend, D. Tommasi, J.A. Smith, S. Koenigstein, M. Weijerman and J.
Link. Accepted May 2021. Management Strategy Evaluation: Allowing the
Light on the Hill to Illuminate More than One Species. Original
Research, Front. Mar.~Sci. - Marine Fisheries, Aquaculture and Living
Resources.

Staudinger, M., A. Lynch, S. Gaichas, M. Fox, D. Gibson-Reinemer, J.
Langan, A. Teffer, S. Thackeray, and I. Winfield. Accepted April 2021.
How does climate change affect emergent properties of aquatic
ecosystems? Fisheries.

Kritzer, J.P., T. Yi, C. Yong, C. Costello, S. Gaichas, T. Nies, E.
Penas,K. Sainsbury, S. Changchun, C. Szuwalski, and Z. Wenbin. Advancing
multispecies management in China: lessons from international experience.
In review at Aquaculture and Fisheries.

\hypertarget{other-papers-book-chapters}{%
\subsection{Other papers (book
chapters)}\label{other-papers-book-chapters}}

Gaichas, S.K., C. Reiss, and M. Koen-Alonso. 2014. Ecosystem Based
Management in high latitude ecosystems. Chapter 10, p.~277-324 in The
Sea, Volume 16: Marine Ecosystem Based Management. Harvard U. Press.

\hypertarget{reports}{%
\subsection{Reports}\label{reports}}

\emph{Northeast Fisheries Science Center State of the Ecosystem Reports;
lead editor for Mid-Atlantic 2017-2021, lead editor for New England
2017-2020, supporting editor for New England 2021}

2021 State of the Ecosystem:

\begin{itemize}
\tightlist
\item
  Mid-Atlantic
  \url{https://apps-nefsc.fisheries.noaa.gov/rcb/publications/soe/SOE_MAFMC_2021_Final-revised.pdf}
\item
  New England
  \url{https://apps-nefsc.fisheries.noaa.gov/rcb/publications/soe/SOE_NEFMC_2021_Final-revised.pdf}
\end{itemize}

2020 State of the Ecosystem:

\begin{itemize}
\tightlist
\item
  Mid-Atlantic \url{https://doi.org/10.25923/1f8j-d564}
\item
  New England \url{https://doi.org/10.25923/4tdk-eg57}
\end{itemize}

2019 State of the Ecosystem:

\begin{itemize}
\tightlist
\item
  Mid-Atlantic
  \url{https://www.mafmc.org/s/Tab09_State-of-the-Ecosystem-Report_2019-04.pdf}
\item
  New England
  \url{https://s3.amazonaws.com/nefmc.org/11_SOE-NEFMC-2019.pdf}
\end{itemize}

2018 State of the Ecosystem:

\begin{itemize}
\tightlist
\item
  Mid-Atlantic
  \url{https://www.mafmc.org/s/Tab04_State-of-the-Ecosystem-Apr2018.pdf}
\item
  New England
  \url{https://s3.amazonaws.com/nefmc.org/2_Ecosystem-Status-Report.pdf}
\end{itemize}

2017 State of the Ecosystem:

\begin{itemize}
\tightlist
\item
  Mid-Atlantic
  \url{https://www.mafmc.org/s/Tab02_2017-04_State-of-the-Ecosystem-and-EAFM.pdf}
\item
  New England:
  \url{https://s3.amazonaws.com/nefmc.org/2_2016-State-of-the-Ecosystem-Report.pdf}
\end{itemize}

\emph{Recent International Council for the Exploration of the Seas
(ICES) Expert Group reports}

ICES. 2021. Working Group on Multispecies Assessment Methods (WGSAM;
outputs from 2020 meeting). ICES Scientific Reports. 3:10. 231
pp.~\url{https://doi.org/10.17895/ices.pub.7695}

ICES. 2021. Workshop of Fisheries Management Reference Points in a
Changing Environment (WKRPChange, outputs from 2020 meeting). ICES
Scientific Reports. 3:6. 39 pp.https://doi.org/10.17895/ices.pub.7660

ICES. 2019. Working Group on Multispecies Assessment Methods (WGSAM).
ICES Scientific Reports.1:91. 320
pp.~\url{http://doi.org/10.17895/ices.pub.5758}

ICES. 2018. Report of the Working Group on Multispecies Assessment
Methods (WGSAM), 15--19 October 2018, Paris, France. ICES CM
2018/HAPISG:20. 89 pp.

\emph{Selectied NOAA Technical Memoranda}

Townsend, H., K. Aydin, K. Holsman, C. Harvey, I. Kaplan, E. Hazen, P.
Woodworth-Jefcoats, M. Weijerman, T. Kellison, S. Gaichas, K. Osgood, J.
Link (editors). 2017. Report of the 4th National Ecosystem Modeling
Workshop (NEMoW 4): Using Ecosystem Models to Evaluate Inevitable
Trade-offs. U.S. Dept. of Commerce, NOAA Tech. Memo. NMFS-F/SPO-173, 77
p.

Link, J.S., D. Mason, T. Lederhouse, S. Gaichas, T. Hartley, J. Ianelli,
R. Methot, C. Stock, C. Stow, and H. Townsend. 2015. Report from the
Joint OAR-NMFS Modeling Uncertainty Workshop. U.S. Dept. of Commer.,
NOAA. NOAA Technical Memorandum NMFS-F/SPO-153, 31 p.

Link, J. S., T.F. Ihde, H. Townsend, K. Osgood, M. Schirripa, D.
Kobayashi, S. Gaichas, J. Field, P. Levin, K. Aydin, G. Watters, and C.
Harvey (editors). 2010. Report of the 2nd National Ecosystem Modeling
Workshop (NEMoW II): Bridging The Credibility Gap -- Dealing With
Uncertainty In Ecosystem Models. U.S. Dep. Commerce, NOAA Tech. Memo.
NMFS-F/SPO-102, 72 p.

Conners, M.E., J. Cahalan, S. Gaichas, W.A. Karp, T. Loomis, and J.
Watson. 2009. Sampling for estimation of catch composition in Bering Sea
trawl fisheries. NOAA Tech. Memo. NMFS-AFSC-199, 42 p.~

Aydin, K., S. Gaichas, I. Ortiz, D. Kinzey, and N. Friday. 2007. A
comparison of the Bering Sea, Gulf of Alaska, and Aleutian Islands large
marine ecosystems through food web modeling. U.S. Dep. Commer., NOAA
Tech. Memo. NMFS-AFSC-178, 298 p.

Gaichas, S.K., and J.C. Field, 2003. Comparative modeling approaches for
exploited marine ecosystems in the North Pacific: what are these models
good for? In Mace, P.M. (ed.), Proceedings of the seventh NMFS National
Stock Assessment Workshop, (Re)building Sustainable Fisheries and Marine
Ecosystems. NOAA Tech. Memo NMFS-F/SPO-62.

\hypertarget{others-fishery-management-council-documents}{%
\subsection{Others (Fishery Management Council
Documents)}\label{others-fishery-management-council-documents}}

April 2020-2021: Lead author, State of the Ecosystem Request tracking
memo for the Mid-Atlantic and New England Fishery Management Councils.
\url{https://www.mafmc.org/s/Tab01_2020-SOE-Report_2020-04.pdf} (2020;
pp 33-49 of combined pdf),
\url{https://www.mafmc.org/s/Tab06_SOE-Mid-Atlantic-Risk-Assessment__2021-04.pdf}
(2021; pp 44-60 of combined pdf)

October 2017-2021: Lead author, Mid-Atlantic Fishery Management Council
Ecosystem Approach to Fisheries Management Risk Assessments.
\url{https://www.mafmc.org/s/Tab08_EAFM-Risk-Assessment.pdf} (2017),
\url{https://www.mafmc.org/s/MAB_RiskAssess_08_18.pdf} (2018),
\url{https://www.mafmc.org/s/Tab10_EAFM_Updates_2019-04.pdf} (2019),
\url{https://www.mafmc.org/s/Tab01_2020-SOE-Report_2020-04.pdf} (2020;
pp 50-62 of combined pdf),
\url{https://www.mafmc.org/s/Tab06_SOE-Mid-Atlantic-Risk-Assessment__2021-04.pdf}
(2021; pp 61-81 of combined pdf)

December 2019: Mid-Atlantic Fishery Management Council Ecosystem
Approach to Fisheries Management Conceptual models and visualizations.
\url{https://nefsc.github.io/READ-SSB-DePiper_Summer_Flounder_Conceptual_Models/sfconsmod_riskfactors_subplots.html}
\url{https://nefsc.github.io/READ-SSB-DePiper_Summer_Flounder_Conceptual_Models/sfconsmod_final_2col.html}

April 2016: Contributor to Mid-Atlantic Fishery Management Council
Ecosystem Approach to Fisheries Management Guidance Document (1st
Draft). \url{http://www.mafmc.org/s/Tab05_EAFM.pdf}

February 2016: Lead author for A Framework for Incorporating Species,
Fleet, Habitat, and Climate Interactions into Fishery Management: A
DRAFT White Paper to Inform the Mid-Atlantic Fishery Management Council.
\url{http://www.mafmc.org/s/02_DRAFTInteractionsWhitePaper_bbdraft.pdf}

April 2015: Contributor/Editor for Climate Change and Variability: A
White Paper to Inform the Mid-Atlantic Fishery Management Council on the
Impact of Climate Change on Fishery Science and Management.
\url{http://www.mafmc.org/s/Climate-Change-and-VariabilityWhite-Paper_BBook_second_draft_final.pdf}

December 2014: Contributor to Managing Forage Fishes in the Mid-Atlantic
Region: A White Paper to Inform the Mid-Atlantic Fishery Management
Council.
\url{http://www.mafmc.org/s/3_ForageWhitePaper_Nov2014_draft3-2.pdf}

November 2010: Lead author, Bering Sea Ecosystem Assessment, and
Ecosystem Considerations SAFE co-editor:
\url{http://access.afsc.noaa.gov/reem/ecoweb/Eco2010.pdf}

October 2010: GOA Crab Protection Closures Environmental Assessment,
contributed ecosystem effects analysis in section 4:
\url{http://www.fakr.noaa.gov/npfmc/current_issues/bycatch/GOAcrab.pdf}

September 2010: Ecosystem Considerations SAFE editor (while S. Zador was
on leave):
\url{ftp://ftp.afsc.noaa.gov/afsc/public/Plan_Team/ecosystem.pdf}

November 2009: Ecosystem Assessment coauthor:
\url{http://www.afsc.noaa.gov/refm/docs/2009/ecosystem.pdf}

September 2008: Arctic Fishery Management Plan Environmental Analysis;
contributed Arctic Ecosystem Description, Chapter 8, Section 1:
\url{http://www.fakr.noaa.gov/analyses/arctic/ArcticFMP_EA1108.pdf}

December 2007: Aleutian Islands Fishery Ecosystem Plan; Ecosystem Plan
Team member, contributed food web analyses:
\url{http://www.fakr.noaa.gov/npfmc/current_issues/ecosystem/AIFEPbrochure1207.pdf}

1999-2006: Stock assessment reports (see above under Professional
History) contributed to North Pacific Fishery Management Council Stock
Assessment and Fishery Evaluation Reports:
\url{https://archive.fisheries.noaa.gov/afsc/REFM/stocks/Historic_Assess.htm}

\hypertarget{presentations}{%
\section{Presentations}\label{presentations}}

Summary: Speaker, moderator, and symposium organizer at regional to
international professional meetings (American Fisheries Society, ASLO,
Conservation Biology, Ecological Society of America, GLOBEC, IMCC, World
Fisheries Congress, ICES Annual Science Conference) 1995-present.

\hypertarget{talks-presented-at-professional-meetings}{%
\subsection{Talks presented at professional
meetings}\label{talks-presented-at-professional-meetings}}

\emph{Upcoming invited presentations}

Invited keynote speaker on best practices in multispecies modeling,
Multispecies modeling applications workshop (online), June 2021.

Invited keynote speaker on ecosystem indicators contextualizing stock
assessments, U.S. Fishery Management Council 7th Scientific Coordination
Subcommittee (formerly National SSC), August 2020 (meeting postponed to
2022)

Invited speaker for two American Fisheries Society (AFS) Annual meeting
symposia, November 2021. Submitted abstracts:

\begin{itemize}
\item
  Gaichas, S., B. Muffley, G. DePiper. Climate news you can use: risk
  assessment and ecosystem reporting in the US Mid-Atlantic. (Connecting
  the dots, management approaches for climate-ready fisheries symposium)
\item
  Gaichas, S., V. Saba, S. Lucey. Integrated Ecosystem Assessment and
  Ecological Forecasting: Let's Do This. (Ecological Forecasting:
  Bridging fish, ecosystems, weather, water, climate and communities
  symposium)
\end{itemize}

\emph{Selected presentations (post PhD)}

Gaichas, S.K., G. DePiper, R. Seagraves, B. Muffley, S. Lucey. There is
no I in EAFM: adapting Integrated Ecosystem Assessment for Mid-Atlantic
Fisheries Management. 2019 Coastal and Estuarine Research Foundation
Biennial Conference, Mobile AL. (Presented by Sean Lucey due to conflict
with review panel)

Gaichas, S.K., J. Deroba, M-Y. Lee. Are any herring harvest control
rules good for both fisheries and predators? 2018 ICES Annual Science
Conference, Hamburg, Germany.

Deroba J.J., S.K. Gaichas (speaker), M‐Y. Lee, R.G. Feeney, D. Boelke,
B. Irwin. Getting on the same page, or at least in the same library:
lessons in communication from a stakeholder-driven MSE for Northeast US
Atlantic herring. 2017 IMBER IMBIZO, Woods Hole MA.

Gaichas, S.K., R. Seagraves, J. Coakley, G. DePiper, V.G. Guida, J.
Hare, P. Rago and M. Wilberg. The Mid-Atlantic Fishery Management
Council's Framework for Incorporating Species, Fleet, Habitat, and
Climate Interactions into Fishery Management: Initial Risk Assessment.
2017 AFS Annual Meeting, Tampa FL.

(Invited) Gaichas, S., M. Fogarty, G. DePiper, G. Fay, R. Gamble, S.
Lucey, and L. Smith. Combining stock, multispecies, and ecosystem level
status determination criteria: what tradeoffs can we expect? 2015 ICES
symposium ``Targets and limits for long term fisheries management,''
Athens, Greece.

Gaichas, S.K., M. Fogarty, R.J. Gamble, S.M. Lucey, L. Smith, C.
Perretti, G. Fay. Multispecies stock assessment for Georges Bank: model
development, performance testing, and multimodel inference. 2015 AFS
Annual Meeting, Portland, OR.

(Invited) Gaichas, S., J. Link. Incorporating Climate into Ecosystem
Based Fisheries Management. 2015 NOAA GFDL Application of Seasonal to
Decadal Climate Predictions for Marine Resource Management Workshop,
Princeton, NJ.

Gaichas, S., J. Link, J. Hare, M. Fogarty, P. Pinto da Silva, G.
DePiper, J. Cournane, J. Manderson, S. Lucey, M. Lowe, S. Large, and G.
Fay. Towards operational assessments: selection, vetting, and
standardized analysis of ecosystem indicators for the Northeast US Large
Marine Ecosystem. 2014 ICES Annual Science Conference, A Coruña, Spain.

Gaichas, S., J. Link, T. Miller, T. Essington, A. Bundy, J. Boldt, K.
Drinkwater, and E. Moksness. What drives marine fisheries production?
Emergent features from comparisons across 13 northern hemisphere
ecosystems. 2012 World Fisheries Congress, Edinburgh, Scotland.

Gaichas, S., K. Aydin, S. Zador, and I. Ortiz. Wasp waist or beer belly?
Modeling food web structure and energetic control in Alaskan marine
ecosystems, with implications for fishing and environmental forcing.
2012 AIFRB Symposium, New Bedford, MA.

Gaichas, SK, MJ Fogarty, L Col, G Fay, R Gamble, S Large, JS Link, S
Lucey and TJ Miller. Developing a multispecies model for ecosystem based
management on the Northeast U.S. continental shelf. 2012 Ecologial
Society of America Annual meeting, Portland OR.

(Invited) Gaichas, S. Uncertainty in ecosystem indicators: known knowns,
known unknowns, and unknown unknowns. 2011 PICES FUTURE workshop on
Ecosystem Indicators, Honolulu HI.

Fogarty, M., S. Gaichas (presenter), H. Benoit, T. Essington, C. Fu, R.
Gambple, M. Koen-Alonso, J. Link. Assembly Rules for Aggregate-Species
Production Models: Simulations in Support of Management Strategy
Evaluation. 2011 AFS Annual Meeting, Seattle, WA.

Gaichas, S., K. Aydin, and R. Francis. What Drives Dynamics in the Gulf
of Alaska? Integrating Hypotheses of Species, Fishing, and Climate
Relationships Using Ecosystem Modeling. 2010 Lowell Wakefield Ecosystems
Symposium, Anchorage AK.

Gaichas, S. From the Aleutians to the Arctic: integrating ecosystem
approaches within the Alaska fishery management process. 2010 AFS Annual
Meeting,n Pittsburgh, PA.

Gaichas, S., K. Aydin, G. Odell, and R. Francis. Complex systems models
of trophic and fishing interactions with implications for research and
management. 2009 GLOBEC Open Science Meeting, Victoria BC.

Gaichas, S. K. Aydin, G. Odell, and R. Francis. Identifying critical
interactions and thresholds in eastern north Pacific ecosystems through
model simulations. 2008 Advances in Marine Ecosystem Modelling Research
(AMEMR) symposium, Plymouth, UK.

Gaichas, S., G. Odell, R. Francis, and K. Aydin. Fishing the Gulf of
Alaska marine food web: do predator prey interactions imply ecosystem
thresholds? 2007 PICES Annual meeting, Victoria, B.C. Also presented at
2007 AFS Annual Meeting, San Francisco CA.

(Invited) Gaichas, S., G. Odell, R. Francis, K. Aydin. Fishing the Gulf
of Alaska marine food web: are there thresholds related to functional
response? 2007 Society for Conservation Biology, Port Elizabeth, South
Africa.

Gaichas, S. K. Aydin, and V. Agostini, Quantitative methods for
comparative ecosystem analysis: relationships and thresholds in the Gulf
of Alaska and the California Current. 2006 PICES Annual meeting,
Yokohama ,Japan.

\hypertarget{posters-presented-at-professional-meetings-recent}{%
\subsection{Posters presented at professional meetings
(recent)}\label{posters-presented-at-professional-meetings-recent}}

Gaichas, S., G. DePiper, R. Seagraves, L. Colburn, A. Loftus, M. Sabo,
B. Muffley. Which MSE should you do first? Prioritizing MSE investments
using risk assessment under an Ecosystem Approach to Fishery Management.
National SSC meeting in San Diego, January 2018

Deroba, J., S. Gaichas, Lee M‐Y., Feeney R.G., Boelke D., Irwin
B.``Getting on the same page, or at least in the same library: lessons
in communication from a stakeholder-driven management strategy
evaluation for northeast US Atlantic herring. National SSC meeting in
San Diego, January 2018

\hypertarget{other-seminarspresentations}{%
\subsection{Other
seminars/presentations}\label{other-seminarspresentations}}

Summary: Regular presentations to managers and stakeholders at Fishery
Management Council meetings and management strategy evaluation
stakeholder workshops. Presenter for the Marine Resource Education
Program and at the Maine Fishermen's Forum. Presenter in multiple agency
and academic seminar series including OneNOAA, Alaska and Northeast
Fishery Science Centers, University of Washington, University of
Massachusetts, Rutgers University, and others. Invited speaker for
public lecture series, teacher workshops, outreach events, and
university courses. Links to representative presentations within the
past 3 years are listed, as well as titles of selected older
presentations.

\textbf{2021}

May: (Invited) ``Integrated Ecosystem Assessment (IEA) and State of the
Ecosystem Reporting,'' for the Mid-Atlantic Council on the Ocean Ocean
Forum;
\url{https://noaa-edab.github.io/presentations/20210506_MACO_IEASOE_Gaichas.html}

April: ``State of the Ecosystem and EAFM Risk Assessment:
Mid-Atlantic,'' for the MAFMC;
\url{https://noaa-edab.github.io/presentations/20210407_MAFMC_Gaichas.html}

March-April: ``Mid-Atlantic Ecosystem Approach,'' for Mid-Atlantic
Fishery Management Council Summer Flounder MSE stakeholder workshops
(3);
\url{https://noaa-edab.github.io/presentations/20210329_MAFMC_MSEwkshp_EAFM_Intro_Gaichas.html}

March: (Invited) ``How is our ocean doing? Data, statistics, and
modeling for management advice,'' guest lecture for Stockton University
Survey of Marine Life course;
\url{https://noaa-edab.github.io/presentations/20210329_Stockton_ecotalk_Gaichas.html}

March: ``State of the Ecosystem: Mid-Atlantic,'' for the MAFMC SSC;
\url{https://noaa-edab.github.io/presentations/20210407_MAFMC_Gaichas.html}

February: (Invited) ``Developing and using ecosystem information in
fishery management: Examples from the Northeast US,'' for the Department
of Fisheries and Oceans Canada ecosystems working group;
\url{https://noaa-edab.github.io/presentations/20210209_DFO_EBFM_Gaichas.html}

January: (Invited) ``Mid-Atlantic State of the Ecosystem Report,'' for
Chesapeake Bay Program Goal Implementation Team Meeting;
\url{https://noaa-edab.github.io/presentations/20210114_ChesapeakeFishGIT_Gaichas.html}

\textbf{2020}

November: ``Integrating Climate into SOE and Model Skill Assessment,''
for Northeast Fisheries-Climate PI workshop;
\url{https://docs.google.com/presentation/d/184Af-afIIv-aGSRT6-Ed9SRmqf5n8YOO/edit\#slide=id.p1}

October: ``Multispecies models: US update,'' for ICES Working Group on
Multispecies Assessments;
\url{https://noaa-edab.github.io/presentations/20201012_WGSAM_USupdate_Gaichas.html}

October: (Invited) ``Species Interactions in Multispecies models,'' for
NOAA Fisheries Integrated Toolbox Technical meeting;
\url{https://noaa-edab.github.io/presentations/20201008_FIT_MSmods_Gaichas.html}

September: (Invited) ``Herring HCR accounting for its role as forage,''
for ICES Workshop on Reference Point Changes;
\url{https://noaa-edab.github.io/presentations/20200923_herringWKRPchange_Gaichas.html\#1}

September: ``Mid-Atlantic Ecosystem Approach, Overview and Progress,''
for MAFMC MSE introductory workshop;
\url{https://noaa-edab.github.io/presentations/20200504_MAFMC_EAFM_Intro_Gaichas.html}

September: (Invited) ``US Reference Points Introduction,'' for ICES
Workshop on Reference Point Changes;
\url{https://noaa-edab.github.io/presentations/20200921_USrefpts_WKRPchange_Gaichas.html}

September: ``State of the Ecosystem: Improving Synthesis and Management
Applications'' and ``wind energy discussion,'' for MAFMC SSC;
\url{https://noaa-edab.github.io/presentations/20200909_MAFMC_SSC_Gaichas.html}
and
\url{https://noaa-edab.github.io/presentations/20200909_MAFMC_SSC_WindSOE_Gaichas.html}

August: ``FAIR data: resources and details! Avoiding data problems, best
practices for metadata and controlled vocabularies,'' for EDAB seaside
chat series;
\url{https://noaa-edab.github.io/presentations/20200803_seasideFAIRdata_Gaichas.html}

July: (Invited) Deep Uncertainty and MSE, for NOAA Science Advisory
Board's Ecosystem Sciences and Management Working Group;
\url{https://noaa-edab.github.io/presentations/20200716_MSEUncertainty_Gaichas.html}

June: ``Open Science, FAIR data, and the Fish Condition project,'' for
EDAB seaside chat series;
\url{https://noaa-edab.github.io/presentations/20200601_seasideFAIRdata_Gaichas.html}

May: Climate and Fisheries, for National Academy of Sciences Ocean
Studies Board panel; \url{https://noaa-}
edab.github.io/presentations/20200522\_NASOSB\_Gaichas.html

May: ``State of the Ecosystem 2020,'' for ICES Working Grouo on the
Northwest Atlantic Regional Sea, (pre-recorded in April);
\url{https://noaa-edab.github.io/presentations/20200508_US_SOEUpdate_Gaichas.html}

May: ``US Council update (EAFM/EBFM),'' for ICES Working Grouo on the
Northwest Atlantic Regional Sea, (pre-recorded in April);
\url{https://noaa-edab.github.io/presentations/20200508_USCouncilUpdate_Gaichas.html}

April: (Invited) ``What does math have to do with fish? Data,
statistics, and modeling for management advice,'' Worchester Polytechnic
Institute SIAM seminar;
\url{https://noaa-edab.github.io/presentations/20200428_WPI_SIAMtalk_Gaichas.html}

April: ``State of the Ecosystem: Mid-Atlantic 2020 with updates to EAFM
Risk Assessment'' for MAFMC;
\url{https://noaa-edab.github.io/presentations/20200407_MAFMC_Gaichas.html}

March: State of the Ecosystem: New England" for the NEFMC SSC;
\url{https://noaa-edab.github.io/presentations/20200331_NEFMC_SSC_Gaichas.html}

March: ``State of the Ecosystem: Mid-Atlantic 2020 with updates to EAFM
Risk Assessment'' for the MAFMC SSC in Baltimore MD;
\url{https://noaa-edab.github.io/presentations/20200310_MAFMC_SSC_Gaichas.html}

February: ``What is Ecosystem Based Fishery Management?'' for Marine
Resource Education Program Science Workshop in Falmouth MA;
\url{https://noaa-edab.github.io/presentations/20200226_EBFM_MREP_Gaichas.html}

January: (Invited) ``There is no I in EAFM: Adapting Integrated
Ecosystem Assessment for Mid-Atlantic Fisheries Management,'' for NOAA
EBM/EBFM webinar series (online);
\url{https://noaa-edab.github.io/presentations/20200108_EBFM_Gaichas.html}
recording: \url{https://youtu.be/VnI3XMRw3FI}

\textbf{2019}

December: ``Summer Flounder Conceptual Model and Questions,''
interactive webpage presentations for MAFMC, Annapolis MD;
\url{https://nefsc.github.io/READ-SSB-DePiper_Summer_Flounder_Conceptual_Models/sfconsmod_final_2col.html}

November: (Invited) ``Progress on Implementing Ecosystem-Based Fisheries
Management in the United States Through the Use of Ecosystem Models and
Analysis,'' for CAUSES workshop, Woods Hole MA;
\url{https://noaa-edab.github.io/presentations/20191120_CAUSES_Gaichas.html}

November: (Invited) ``Fragile ecosystems, robust assessments? (What I
did with my summer vacation),'' OneNOAA AFSC seminar (online). Also
given as UMass School of Marine Science and Technology in New Bedford
MA, and NEFSC seminar in Woods Hole MA in October;
\url{https://noaa-edab.github.io/presentations/20191009_SMAST_Gaichas.html}

September: ``Conceptual models: What are they? How can we use them?''
for MAFMC Ecosystem and Ocean Planning Committee (EOP), Baltimore MD;
\url{https://noaa-edab.github.io/presentations/20190918_MAFMC_EOP_Gaichas.html}

August: ``Management Strategy Evaluation, National Progress and Needs,''
briefing for NOAA leadership (online)

July: (Invited) ``Ecosystem modelling approaches: What questions can we
address?'' Shoals Marine Lab Integrated Ecosystem Assessment course
guest lecture and lab, Appledore Island ME;
\url{https://noaa-edab.github.io/presentations/20190723_SMLlab_Gaichas.html}

July: (Invited) ``Fishing within food webs: Modeling for management
advice,'' Shoals Marine Lab Rock Talk evening seminar series, Appledore
Island ME;
\url{https://noaa-edab.github.io/presentations/20190723_SMLrocktalk_Gaichas.html}

May: (Invited) ``Generating datasets for model performance testing: How
challenging can it be?'' Institute for Marine Research (IMR)
Departmental seminar, Bergen, Norway;
\url{https://noaa-edab.github.io/presentations/20190528_deptIMR_Gaichas.html}

May: (Invited) ``Implementing an Ecosystem Approach to Fishery
Management: A Risk Assessment Example,'' IMR Institute seminar series,
Bergen, Norway

May: ``Integrated Ecosystem Assessment: Mid-Atlantic Fishery
Management's Success Story'' for NOAA-IMR bilateral meeting, Bergen,
Norway;
\url{https://noaa-edab.github.io/presentations/20190506_IEA_Gaichas.html}

May: ``Scientist Exchange (2019): A collaborative U.S.- IMR research
project'' for NOAA-IMR bilateral meeting, Bergen, Norway;
\url{https://noaa-edab.github.io/presentations/20190507_REDUS_Gaichas.html}

April: ``State of the Ecosystem: Mid-Atlantic 2019'' for the MAFMC;
\url{https://noaa-edab.github.io/presentations/20190410_MAFMC_Gaichas.html}

April: ``EAFM Risk Assessment: Mid-Atlantic 2019 Update,'' for the
MAFMC;
\url{https://noaa-edab.github.io/presentations/20190410_MAFMC_risk_Gaichas.html}

March: ``State of the Ecosystem: Mid-Atlantic 2019 with updates to EAFM
Risk Assessment'' for the MAFMC SSC;
\url{https://noaa-edab.github.io/presentations/20190320_MAFMC_SSC_Gaichas.html}

\textbf{2018}

October: ``The Dream and the Reality: challenges with modeling marine
mammals for the New England Fishery Management Council's herring MSE,''
for Workshop on MSE applioed to Marine Mammal Conservation and
Management, Providence, RI

October: WGSAM

September: (Invited) Climate-models NE workshop

September: MAFMC EOP

July: WPI REU

May: (Invited) presenter, ``Implications of Ecosystem Approaches to
Fisheries Management (EAFM) for Pelagic Fisheries,'' for workshop
co-organized by European pelagics fishing industry and MAFMC,
Gloucester, MA.

April: (Invited) ``Can we integrate ecosystem and multispecies
interactions into TACs and other fishery management?'' for International
TAC workshop in Xiamen China

April: SOE for the MAFMC

March: ``US Risk Assessment and Ecosystem Reporting Update'' for ICES
WGNARS

March: SOE for the MAFMC SSC

February: (Invited) ``Analytical approaches for Ecosystem Based Fishery
Management: Models and Indicators,'' for Marine Resources Education
Program (MREP) 200 course on EBFM; also organizing committee member.
(Invited to parallel West Coast MREP but unable to attend)

\textbf{2017}

December: ``The Dream and the Reality: Meeting decision-making times
frames while incorporating ecosystem and economic models into management
strategy evaluation,'' QUEST webinar

November: (Invited) ``IEAs'' for NSF Wind Energy Workshop in New
Bedford, MA

October: ``Update on US modeling activities,'' ``Indicator based risk
assessment,'' ``Herring MSE update'' and 6 presentations for other US
scientists at ICES WGSAM in San Sebastian, Spain

October: ``Mid-Atlantic Fishery Management Council EAFM Based Risk
Assessment,'' for MAFMC in Riverhead, NY

September: ``Mid-Atlantic EAFM Risk Assessment: Review context, prior
work, Expand to new elements,'' for MAFMC Ecosystem and Ocean Planning
Committee (EOP)

May: ``State of the Ecosystem Report: Northeast US Shelf'' and
``Northeast Climate Vulnerability Assessment and Regional Action Plan,''
for US-Norway bilateral in Woods Hole, MA

April: (Invited) ``State of the Ecosystem Report: Mid-Atlantic,'' for
Fisheries Leadership and Sustainability Forum in Monterey, CA

April: ``State of the Ecosystem Report: Mid-Atlantic,'' for the MAFMC in
Avalon, NJ

March: ``State of the Ecosystem Report: Mid-Atlantic'' for the MAFMC SSC
(online)

March: ``Predator models'' for NEFMC Herring MSE peer review in Boston,
MA

March: ``US IEA Update'' for ICES WGNARS in Halifax, NS

February: ``Northeast Fisheries Science Center Ecosystem Modeling:
Applications Investigating Tradeoffs'' and ``Management Strategy
Evaluation (MSE)'' for National Ecosystem Modeling Workshop in St
Petersburg, FL.

\textbf{2016}

December: ``Predator models'' and ``Predator metrics and results'' for
Herring MSE stakeholder workshop in Portsmouth, NH

October: ``Update on US MSE activities and Atlantis Poseidon Adventure
project,'' Mid-Atlantic Ecosystem Approach to Fisheries Management'',
and two presentations for other US scientists. ICES WGSAM in Reykjavik,
Iceland

June: (Invited) ``Management Strategy Evaluation in the US,'' kickoff
meeting for the Norwegian Research Council project Reducing Uncertainty
in Stock Assessment (REDUS), Bergen, Norway

June: ``Management Strategy Evaluation (MSE)'' for NEFSC Ecosystem and
Climate Science Program Review, Woods Hole, MA

May: ``What do we know about Atlantic herring's role as forage?'' for
NEFMC herring MSE stakeholder workshop in Portland, ME

\textbf{Selected presentations, 2015-2007}

September 2015: (Invited) ``Seasonal forecasting and ecosystem based
fishery management.'' for American Meteorological Society Capitol Hill
Policy Briefing on Applications of Seasonal Forecasting, Washington,
DC\\
April 2015: (Invited) ``Uncertainty in Ecosystem Models: Taxonomy and
Treatment,'' NMFS-OAR Uncertainty Workshop, Ann Arbor, MI

November 2014: ``WGNARS: IEA progress, opportunities, and challenges''
ICES WKRISCO in Copenhagen, Denmark

November 2014: ``Ecosystem metrics for fisheries management: examples
and suggestions,'' for Chesapeake Bay Forage Workshop in Solomons, MD

October 2014: ``Mixed fisheries in multispecies and ecosystem models''
for ICES WGSAM (presented remotely)

February 2013: (Invited) ``Visualizing complexity'' Data Visualization
Workshop, Portland ME

April 2011: ``Diet data products for stock assessment'' for the
Stock-specific ecosystem indicators workshop hosted by Alaska Fisheries
Science Center, Seattle, WA

March 2011: ``Central GOA dynamic model simulations of whale
restoration, jellyfish fluctuations, and no fishing'' for GLOBEC
Pan-Regional Synthesis PI meeting, Seattle, WA

December 2010: ``Marine ecosystems science and management from Alaska to
Puget Sound'' with Chris Harvey (NWFSC) at the Science Cafe, an informal
public lecture series in Seattle, WA sponsored by Pacific Science Center

September 2010: ``Ecosystem Considerations SAFE'' to the joint
Groundfish Plan Teams, Seattle, WA

May 2010: ``Arctic ecosystem knowledge, where do we stand and where do
we go from here?'' in the AFSC seminar series, Seattle, WA

May 2010: ``GOA ecosystem overview and data summary,'' and examples of
fitting our models to time series data at CAMEO stock production
modeling workshop, Woods Hole MA, May 10-14

April 2010: ``Fishing policy simulations in end-to-end models'' at CAMEO
end to end modeling workshop, Woods Hole MA

March 2010: ``GOA food web modeling and management applications'' for
CAMEO Predator-Prey Interactions workshop in Corvallis, OR

March 2010: ``Ecosystem Considerations in the North Pacific: Groundfish
SAFE \&Aleutian Islands FEP'' for the Ecosystem Science and Management
workshop in Silver Spring, MD

December 2009: ``Combining single species and ecosystem modeling results
in ecosystem assessments'' for UW-NMFS Fisheries Think Tank on Dec 8,
2009

December 2009: (Invited) ``Where does ecosystem modeling fit into
fisheries management? Examples from the North Pacific'' for UW SAFS
``Young Investigator'' departmental seminar, Seattle, WA.

August 2009: ``Gulf of Alaska food web modeling and predation mortality
estimates: information for single-species M'' at the national workshop
on Natural Mortality, Seattle, WA.

August 2009: ``Analysis of comparisons of single species and ecosystem
model results across NMFS Science Centers'' at NEMoW II meeting in
Annapolis, MD.

January 2008: Unversity of Washington Career Week panelist for marine
science careers, Seattle, WA

May 2008: ``Science on Tap'' informal evening outreach presentation on
ecosystem modeling and fishery management, sponsored by Pacific Science
Center, Seattle, WA

May 2008: Food-web ecology demonstration, Marine Science Weekend at
Pacific Science Center, Seattle WA

May 2008: ``Food web modeling and results,'' Guest lecture for Ray
Hilborn's University of Washington course, Seattle, WA

October 2007: (Invited) ``Alaskan fisheries and deep-sea resources'' for
Asia-Pacific Economic Cooperation (APEC) workshop on developing a
network on deep sea resources and fisheries, in Lima, Peru

October 2007: ``Ecosystem threshold modeling'' at UW mini-workshop,
Seattle WA

October 2007: ``Ecosystem modeling'' for Pacific Science Center Teacher
workshop, Seattle WA

August 2007: ``Ecosim model'' for National Ecosystem Modeling workshop,
Santa Cruz CA

August 2007: ``Computation involved in ecosystem modeling'' for visiting
junior college teachers from Fresno City College and other schools
touring the NOAA facility, Seattle, WA

March 2007: ``GOA ecosystem description and comparison methods used in
Alaska'' for Marine Ecosystems of Norway and the US (MENU) workshop,
Bergen Norway

February 2007: Co-leader of UW Mini-workshop on fitting ecosystem models
to data, Seattle, WA

\hypertarget{funded-research}{%
\section{Funded research}\label{funded-research}}

PI and Co-PI on multiple Northeast Fisheries Science Center internal
R\&D proposals:

\begin{itemize}
\item
  Evaluating environmental indicators of energy density of forage
  species in the Northeast U.S. (2020, 2021)
\item
  Building operational multispecies assessments: comparing performance
  of length- and age-based multi-species modeling approaches (2020)
\item
  Development of oceanographic indicators for Northern shortfin squid
  (2020)
\item
  Climate-driven Atlantis ecosystem model hindcasts and projections for
  the U.S. Northeast Shelf: Implications of future management strategies
  under climate change (2019)
\end{itemize}

Co-PI on multiple Northeast Fisheries Science Center internal
Groundfish/Climate proposals:

\begin{itemize}
\item
  Evaluating annual variation and environmental regulation of fecundity
  in winter and yellowtail flounder with implications for stock
  reference points (2021)
\item
  Incorporating multivariate physical and biological data as covariates
  in state-space age-structured stock assessments (2020)
\end{itemize}

PI and Co-PI on NMFS Sustainable Fisheries National Catch Share Program
and Magnuson-Stevens Act mplementation Temp Funds:

\begin{itemize}
\item
  Mid-Atlantic Fishery Management Council Ecosystem Approach
  Implementation: Management Strategy Evaluation for the Summer Flounder
  Fishery-Social-Ecological System (2021)
\item
  Mid-Atlantic Fishery Management Council Ecosystem Approach
  Implementation: Management Strategy Evaluation for the Summer Flounder
  Fishery-Social-Ecological System (2020)
\end{itemize}

Co-PI on NMFS SF International fellowship (2019): Fragile ecosystems,
fragile assessments: the California Current and Nordic and Barents Seas
under climate change

Co-PI on COCA proposal with GMRI/SMAST on MSE for groundfish under
climate change (2017-present), developing MSE framework and testing a
range of standard and climate-informed harvest control rules for
multiple groundfish species. Postdoc hiring committee member: reviewed
resumes, participated in interviews, selected candidates. Facilitated
collaboration between COCA postdocs and NEFSC scientists, adding
economists.

Co-PI on NMFS Stock Assessment Methods R\&D proposal: A comparison of
length- and age-based multi- and single-species modeling approaches
(2016)

Co-PI on modeling portion of the North Pacific Research Board's GOA
Integrated Ecosystem Research Plan (2010-2011).

Investigator with 22 others on the interdisciplinary ``Collaborative
Research: GLOBEC Pan-Regional Synthesis: End-to-end Energy Budgets in
US-GLOBEC Regions'' project comparing four marine ecosystems in the
Southern Ocean, in the Pacific off Alaska and the U.S. West Coast, and
in the Atlantic off the U.S East Coast (2008-2011).

\hypertarget{professional-affiliationsactivitiesservice}{%
\section{Professional
Affiliations/Activities/Service}\label{professional-affiliationsactivitiesservice}}

\hypertarget{professional-societies}{%
\subsection{Professional societies}\label{professional-societies}}

Current member: American Fisheries Society
(\href{https://fisheries.org/}{AFS}), American Institute of Fishery
Research Biologists (\href{https://www.aifrb.org/}{AIFRB})

Past member: Ecological Society of America, The Society for Conservation
Biology, American Society of Ichthyologists and Herpetologists

\hypertarget{editorial-positions}{%
\subsection{Editorial positions}\label{editorial-positions}}

Guest editor, 2018-2019. Canadian Journal of Fisheries and Aquatic
Sciences virtual special issue ``Under pressure: addressing fisheries
challenges with management strategy evaluation.''

Guest editor, 2011-2012. Marine Ecology Progress Series Theme Section
``Comparative analysis of marine fisheries production.''

\hypertarget{membership-on-panels}{%
\subsection{Membership on panels}\label{membership-on-panels}}

Chaired multispecies model key run reviews for ICES WGSAM, October 2020
(North Sea), October 2019 (Baltic Sea). Participant in WGSAM model key
run reviews 2015-2018.

Invited review panelist:

\begin{itemize}
\item
  Atlantic States Marine Fisheries Commission Atlantic Menhaden stock
  assessment and ecological reference points, November 2019.
\item
  South African Assessment review, November 2018. MSE-based management
  procedures for 4 species were reviewed over the course of a week by an
  international panel. (Also invited for 2019 panel, but declined due to
  conflicts).
\item
  International Whaling Commission, JARPN II February 2016 (Tokyo,
  Japan), NEWREP-NP February 2017 (by correspondence)
\item
  NOAA NMFS Ecosystem Science Program reviews:

  \begin{itemize}
  \tightlist
  \item
    Alaska Program (Juneau, May 2016)
  \item
    Northwest Program (Seattle, July 2016)
  \end{itemize}
\item
  Woods Hole Oceanographic Institution CINAR Fellows review panel,
  December 2012
\item
  External reviewer for the Canada Maritimes Region Ecosystem Research
  Initiative, March 2012
\end{itemize}

\hypertarget{advisory-services}{%
\subsection{Advisory services}\label{advisory-services}}

\begin{verbatim}
Active
\end{verbatim}

\begin{itemize}
\item
  Mid-Atlantic Fishery Management Council Scientific and Statistical
  Committee (SSC), (2014-present): determining Allowable Biological
  Catch (ABC) advice for fishery management and other scientific
  support. Lead for blueline tilefish and ecosystems. Member of OFL-CV
  working group (2017-2019), economics working group (2020-2021), and
  newly-formed ecosystem working group (2021-).
\item
  Mid-Atlantic Fishery Management Council Ecosystem Approach to
  Fisheries Management (EAFM) work group (2013-present): scientific
  support for ongoing EAFM efforts. Presented in Council EAFM
  development workshops (2013-2015), supported development of EAFM
  policy guidance (2016), EAFM risk assessment (2017-2018), EAFM
  conceptual model (2019), and ongping EAFM Management Strategy
  Evaluation (MSE; 2020-).
\item
  ICES Working Group on North Atlantic Regional Seas (WGNARS),
  (2012-present, co-chair 2013-2016): developing ecosystem indicators
  and integrated ecosystem assessments for Canadian and U.S. waters.
  Co-chaired working group meetings in January 2013, February 2014 (also
  local host), and February 2015:

  \begin{itemize}
  \tightlist
  \item
    Developed agendas, 3 year Terms of Reference for 2014-2016,
    co-authored and submitted reports
  \item
    Organized working sessions - 2013: ecosystem indicator
    thresholds/performance testing and ecological risk assessment, led
    section on risk assessment - 2014: reviewing, selecting and
    operationalizing management objectives and large scale ecosystem
    drivers - 2015: conceptual model development for Gulf of
    Maine/Georges Bank connecting large scale ecosystem and social
    drivers, focal ecosystem components, human activities, and human
    social system components as well as selected management objectives
    drawn from US law and published work
  \end{itemize}
\item
  ICES Working Group on Multispecies Assessment Methods (WGSAM),
  (2013-present, co-chair 2017-present): reviewing and contributing to
  multispecies and ecosystem modeling progress across ICES regions.
  Co-chaired working group meetings in October 2017, October 2018,
  October 2019, October 2020:

  \begin{itemize}
  \tightlist
  \item
    Developed agendas, 3 year Terms of Reference for 2019-2021,
    co-authored and submitted reports
  \item
    Developed review criteria and organized working sessions on
    multispecies model key-run reviews
  \item
    Currently leading intersessional work on multispecies model skill
    assessment
  \end{itemize}
\item
  Advisory partner in the European Commission's PANDORA project
  (2018-2022), which is designing toolsets for new dynamic ocean
  resource assessments and exploitation.

  Completed
\item
  ICES Workshop on Reference Point Change (WKRPchange; 2020)
\item
  Atlantic herring benchmark stock assesment working group (2018)
\item
  Herring MSE steering committee member, New England Fishery Managment
  Council (2016)
\item
  Ecosystem Based Fishery Management Plan Development Team member, New
  England Fishery Management Council (2012-2016)
\item
  Invited speaker/participant, Multispecies Fisheries Management
  Workshop, Fujian, China (2018)
\item
  National Climate Assessment (NCA4; 2016-2018) Ecosystem Chapter author
  contributing to climate effects on trophic interactions and emergent
  properties
\item
  National Integrating Climate Effects into Fisheries Management
  (ICE-FM) working group member
\item
  Stellwagen Bank NMS conceptual modeling workshop facilitator (2017)
\item
  ICES Workshop on Regional Seas Commissions and Integrated Ecosystem
  Assessment Scoping (WKRISCO; 2014)
\item
  Advisory partner in the European Commission's MYFISH project
  (2012-2015), which defined alternative approaches to Maximum
  Sustainable Yield including ecological, social, and economic
  dimensions
\item
  NEFSC Stock Assessment Program Review steering committee member
  (2013-2014). Participated in planning meetings (including Directorate
  briefing) and full review meeting 19-23 May 2014. Developed background
  materials and presentation on how well NEFSC integrates ecosystem and
  climate considerations into stock assessments. Contributed to follow
  up and Center response.
\item
  National Stock Assessment Improvement Plan (SAIP) writing team member
  for NEFSC (2013-2014). Contributed to chapters on incorporating
  ecosystem effects into stock assessments and future of stock
  assessment (including IEAs, multimodel inference, etc.).
\item
  Member of the Aleutian Islands Fishery Ecosystem Plan Team (2007-2011)
  and the Gulf of Alaska Fishery Management Plan Team (2000-2011) for
  North Pacific Fishery Management Council. Co-authored the Aleutian
  Islands Fishery Ecosystem Plan (2007) and regularly co-authored the
  Stock Assessment and Fishery Evaluation introduction and summary for
  the Gulf of Alaska. Authored Ecosystem Description and contributed
  analyses for management alternatives for the Arctic Fishery Management
  Plan (2009).
\item
  Co-organizer of the Bering Sea Ecosystem Synthesis Team (2010), a
  multi-disciplinary group charged with identifying important ecosystem
  indicators for fishery management and writing an eastern Bering Sea
  ecosystem assessment for the North Pacific Fishery Management Council.
\item
  Invited participant in Asia-Pacific Economic Cooperation International
  workshop on developing a network for deep sea research in Lima, Peru
  (2007). Presented information on North Pacific ecosystems, fisheries,
  and management, and helped develop a preliminary framework for an
  information-sharing network.
\end{itemize}

\hypertarget{workshops-organizedchaired-sessions-organizedchaired-at-professional-meetings}{%
\subsection{Workshops organized/chaired; sessions organized/chaired at
professional
meetings}\label{workshops-organizedchaired-sessions-organizedchaired-at-professional-meetings}}

\begin{verbatim}
In Progress
\end{verbatim}

ICES/PICES Symposium Small Pelagic Fish: New Frontiers in Science and
Sustainable Management (planned for Lisbon, Portugal November 2022); S6
Reconciling Ecological Roles and Harvest Goals: Development and Testing
Management Strategies to Enhance Marine Ecosystem Services.
\url{https://meetings.pices.int/meetings/international/2022/pelagic/program\#S6}
Convenors: Sarah Gaichas, Cecilie Hansen, Isaac Kaplan, Richard Nash

World Fisheries Congress (planned for Adelaide, Australia/Virtual
September 2021); The land of plenty: Advances and future directions in
population dynamics modeling to support fishery management. Six
consecutive, non-overlapping sub-sessions are planned, including
(conveners in parentheses): general stock assessment methodology and
statistics (André Punt, James Thorson); spatial models (Daniel Goethel,
Aaron Berger); ecosystem dynamics and climate change (Éva Plagányi,
Caleb Gardner, Sarah Gaichas); data-limited methods (Kristen Omori,
Jason Cope, Natalie Dowling); management strategy evaluation and
biological reference points (Carryn de Moor, Richard McGarvey, Ann
Preece); and a summary discussion (Mark Maunder, Patrick Lynch).

\begin{verbatim}
Completed
\end{verbatim}

National Ecosystem Modeling Workshop (NEMoW) 2019 St Petersburg;
Steering committee member and workshop planner -- Developed NEFSC
overview presentation (presented by Scott Large) -- Developed atlantisom
tool demonstration (presented by Christine Stawitz)

American Fisheries Society annual meeting (AFS) 2018 Atlantic City;
Emergent Properties of Aquatic and Marine Ecosystems Due to Climate
Change: An Overview of the Current State of Knowledge. Organizers:
Michelle D. Staudinger, Abigail J. Lynch, and Sarah K. Gaichas

AFS 2017 Tampa; Closing the Loop: Stakeholder Involvement in the
Management Strategy Evaluation (MSE) Process---Advancing Management
Strategy Evaluation. Organizers: Sean M. Lucey, Sarah K. Gaichas, John
F. Walter III, Daniel Goethel, Aaron Berger, and Patrick Lynch

AFS 2017 Tampa; Under Pressure: Defining Harvest Strategies that Account
for Biological, Environmental or Anthropogenic Spatiotemporal
Complexity---Advancing Management Strategy Evaluation. Organizers:
Daniel Goethel, Aaron Berger, Patrick Lynch, Sarah K. Gaichas, John F.
Walter III, and Sean M. Lucey

AFS 2015 Portland, OR; Multispecies models (including humans!)---where
have we been and where do we need to go? Organizers: Sarah K. Gaichas,
Kirstin Holsman, Alan Haynie and Geret DePiper

AFS 2011 Seattle; What Influences Fisheries Production? Comparing the
Effects of Environmental, Fishing, and Food Web Forcing Across Large
Marine Ecosystems. Organizers: Sarah Gaichas, William Stockhausen,
Thomas J. Miller, Jason S. Link, Tim Essington, R. Ian Perry and Alida
Bundy

AFS 2011 Seattle; Ecosystem Modeling: Joint Modeling of Human Behavior
and Fish Populations; Ecosystem Models to Address Fishery Management
Needs. Organizers: Eli Fenichel, Joshua Abbott, Thomas F. Ihde, Sarah K.
Gaichas, Isaac C. Kaplan, Derek M. Orner, Yvonne L. deReynier and Howard
M. Townsend

Comparative Analysis of Marine Ecosystem Organization (CAMEO) Stock
Production Modeling Workshop II Woods Hole, 2011. Organizers: Alida
Bundy, Ken Drinkwater, Tim Essington, Jason Link, Sarah Gaichas, Tom
Miller, Erlend Moksness, and Ian Perry

Society for Conservation Biology's International Marine Conservation
Conference (IMCC) 2009 Washington DC; Bridging the gap between fisheries
and marine conservation to advance ecosystem-based management.
Organizers: Anne Salomon, Olaf Jensen, Vera Agostini, Sarah Gaichas, and
Norm Sloan

AFS 2007 San Francisco; Ecosystem modeling applications in fishery
management. Organizers: Sarah Gaichas, John Field, Howard Townsend,
Megan Tyrell, and Jason Link

\hypertarget{teaching-and-mentoring}{%
\subsection{Teaching and mentoring}\label{teaching-and-mentoring}}

\begin{verbatim}
Graduate-level
\end{verbatim}

\textbf{Adjunct professor} \emph{University of Massachusetts, Dartmouth}

One current College of William and Mary (Virginia Institute of Marine
Science) PhD student committee\\
One current UNH PhD student committee

2021 (Exam) PhD student committee at the University of Massachusetts,
Dartmouth\\
2020 External examiner for a PhD student at University of Iceland\\
2019 (Exam) PhD student committee at the University of Massachusetts,
Dartmouth\\
2019 External examiner for a PhD student at Dalhousie University\\
2016 External examiner for a PhD student at the University of Cape
Town\\
2016 (Exam) PhD student committee at the University of Washington\\
2015 (Exam) PhD student committee at Oregon State University

\begin{verbatim}
Undergraduate-level
\end{verbatim}

IN FISH program 2021 Internship mentor: diet data analysis and
development of ecosystem indicators

Shoals Marine Lab Integrated Ecosystem Research and Management summer
course and lab guest lecturer 2021, 2019: provided ecosystem modeling
background and interactive R shiny app for lab

Invited course and lab guest lecturer for Rutgers 2020 summer course
(postponed due to COVID)

Worcester Polytechnic Institute Research Experience for Undergraduates
in Applied Mathematics, 2018, 2014: organized NEFSC lab visits, R/V
Albatross tour, and provided guidance, background, and data for student
projects

Guest lecturer: Woods Hole Partnership in Education Program (PEP), June
2015; WHOI Fisheries Oceanography Course, April 2014, Rutgers Ecology
and Evolution Dept. Seminar, March 2013, SMAST Fisheries Oceanography
Dept. Seminar, February 2012.

Regular participant in Alaska Fisheries Science Center scientific
education and outreach activities, including Pacific Science Center
programs, Minorities in Marine Science Undergraduate Program
presentations, elementary, secondary, and college educator workshops,
internship application review and placement (2006-2011).

\hypertarget{agency-service}{%
\section{Agency Service}\label{agency-service}}

\hypertarget{committees}{%
\subsection{Committees}\label{committees}}

National Management Strategy Evaluation Working Group co-Chair
(2016-present)

National EBFM workshop participant, July 2020 * Contributed NEFSC info
to ESR presentation * Represented NEFSC in ESR and other discussions

IMR/NOAA bilateral meetings 2019, 2017

Regional IEA Working Group member

(Please see also Advisory Services section for program review support,
national writing teams, etc.)

\hypertarget{other-relevant-information}{%
\section{Other relevant information}\label{other-relevant-information}}

March 2020 - present: considerable time was needed for dependent care
(elementary age kids and elder care), reducing available work hours.
Regularly used the excused absence for dependent care.

\hypertarget{references}{%
\section{References}\label{references}}

Dr.~Vera Agostini\\
Deputy Director\\
Fisheries and Aquaculture Policy and Resources Division\\
Food and Agriculture Organization of the United Nations\\
Viale delle Terme di Caracalla\\
00153 Rome, Italy\\
\href{mailto:Vera.Agostini@fao.org}{\nolinkurl{Vera.Agostini@fao.org}}\\
Work: +39 06570 50183 Fax: +39 06570 53020\\
Skype vera.agostini

Jane DiCosimo\\
Independent Consultant\\
(Retired Chief, NOAA Fisheries Office of Science and Technology
Assessment and Monitoring Division)\\
13411 Norden Dr,\\
Silver Spring, MD 20906 (and)\\
34198 Eaglet Way,\\
Soldotna AK 99669\\
\href{mailto:janedicosimo@gmail.com}{\nolinkurl{janedicosimo@gmail.com}}
Work: +1 907 830-1222

Dr.~Daniel Howell\\
Institute of Marine Research\\
P.O. Box 1870 Nordnes\\
NO-5817 Bergen, Norway\\
\href{mailto:daniel.howell@hi.no}{\nolinkurl{daniel.howell@hi.no}}\\
Work: +47 92063885

Dr.~Jason Link\\
Senior Scientist for Ecosystem Management\\
National Oceanic and Atmospheric Administration\\
National Marine Fisheries Service\\
166 Water St.\\
Woods Hole, MA 02543 USA\\
\href{mailto:jason.link@noaa.gov}{\nolinkurl{jason.link@noaa.gov}}\\
Work: +1 508 495-2340\\
Fax: +1 508 495-2258\\
Cell: +1 774 392-1330

\end{document}
